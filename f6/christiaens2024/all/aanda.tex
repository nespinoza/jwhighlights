%                                                                 aa.dem
% AA vers. 9.1, LaTeX class for Astronomy & Astrophysics
% demonstration file
%                                                       (c) EDP Sciences
%-----------------------------------------------------------------------
%
%\documentclass[referee]{aa} % for a referee version
%\documentclass[onecolumn]{aa} % for a paper on 1 column  
%\documentclass[longauth]{aa} % for the long lists of affiliations 
%\documentclass[letter]{aa} % for the letters 
%\documentclass[bibyear]{aa} % if the references are not structured 
%                              according to the author-year natbib style


\documentclass[longauth]{aa}
%\documentclass[referee]{aa}
\pdfoutput=1

%%%%%%%%%%%%%%%%%%%%%%%%%%%%%%%%%%%%%%%%
\usepackage{txfonts}
\usepackage{xcolor}
\usepackage{graphicx}
\usepackage{array}
\usepackage{amsmath}
\usepackage{float}
\usepackage{hyperref}
\usepackage{ulem}
\usepackage{makecell}
\usepackage{upgreek}

\newcommand{\ikc}[1]{\textcolor{blue}{\textsf{IK: #1 }}}

% Package hyperref Warning: Suppressing link with empty target on input line XXX
% aus https://tex.stackexchange.com/questions/345764/journal-class-shows-package-hyperref-warning-suppressing-link-with-empty-targe
\makeatletter
\renewcommand*\aa@pageof{, page \thepage{} of \pageref*{LastPage}}
%%%%%%%%%%%%%%%%%%%%%%%%%%%%%%%%%%%%%%%%
\bibpunct{(}{)}{;}{a}{}{,} % to follow the A&A style
% New commands to add colored comments
\newcommand{\vc}[1]{#1}
%\newcommand{\vc}[1]{\textcolor{red}{\textbf{#1}}} %other colors: cyan, brown, orange, 
\newcommand{\gp}[1]{\textcolor{teal}{\textbf{GP:} #1}}
%\newcommand{\new}[1]{\textbf{#1}}
\newcommand{\new}[1]{#1}
%\newcommand{\new}[1]{{\leavevmode{\boldmath\bfseries#1}}}

\hypersetup{
    colorlinks,
    linkcolor={red!50!black},
    citecolor={blue!50!black},
    urlcolor={blue!80!black}
}

\DeclareMathOperator*{\argmin}{\arg\!\min}

\begin{document} 
    
    
%\title{Characterization of PDS~70~$b$ and c using JWST/NIRCam}
\title{MINDS: JWST/NIRCam imaging of the protoplanetary disk PDS~70}
\subtitle{A spiral accretion stream and a potential third protoplanet}
\titlerunning{JWST/NIRCam imaging of the protoplanetary disk PDS~70}

% (a) all authors with a significant contribution (in order of their contribution)
\authorrunning{V.~Christiaens et al.}
\author{V.~Christiaens\inst{\ref{KULeuven},\ref{STARLiege}}, %\fnmsep\thanks{F.R.S.-FNRS Postdoctoral Fellow}, 
M.~Samland\inst{\ref{MPIA}}, %additional data reduction
Th.~Henning\inst{\ref{MPIA}}, %PI
B.~Portilla-Revelo\inst{\ref{Kapteyn}}, %radiative transfer models
G.~Perotti\inst{\ref{MPIA}}, % continuous 
E.~Matthews\inst{\ref{MPIA}}, %mass sensitivity curvesfeedback and diagram
O.~Absil\inst{\ref{STARLiege}}, % continuous feedback
L.~Decin\inst{\ref{KULeuven}}, %\\% continuous feedback
I.~Kamp\inst{\ref{Kapteyn}}, % Obs preparation, 
%J.~Bouwman\inst{\ref{MPIA}}, % Obs preparation? No reply? 
%J.~Davies\inst{\ref{MPIA}}, % Advice for data reduction - no reply? 
% useful manuscript feedback
A.~Boccaletti\inst{\ref{LESIA}}, 
B.~Tabone\inst{\ref{SaclayCNRSOrsay}},
%Y.~Aoyama\inst{\ref{Kavli}},
G.-D.~Marleau\inst{\ref{Duisburg},\ref{Tuebingen},\ref{MPIA},\ref{Bern}},
  % for Gabriel, if space permitted: \ref{Duisburg},\ref{Tuebingen},\ref{MPIA},\ref{Bern}
%%%%%%%%%%%%%%%%%%%%%%%%%%%%%%%%%%%%%%%%%
% (b) MIRI co-PIs who are in MINDS alphabetic
%%%%%%%%%%%%%%%%%%%%%%%%%%%%%%%%%%%%%%%%%
E.~F.~van Dishoeck\inst{\ref{Leiden},\ref{MPE}}, M.~G\"udel\inst{\ref{AstroWien},\ref{ETHZ}}, P.-O.~Lagage\inst{\ref{SaclayCEAGif}}, %\\
%C.~Waelkens\inst{\ref{KULeuven}},
%%%%%%%%%%%%%%%%%%%%%%%%%%%
% (c) MINDS GTO team alphabetic
%%%%%%%%%%%%%%%%%%%%%%%%%%%
%A.~Abergel\inst{\ref{SaclayCNRSOrsay}}, I.~Argyriou\inst{\ref{KULeuven}}, 
D.~Barrado\inst{\ref{CABMadrid}}, A.~Caratti o Garatti\inst{\ref{INAFNapoli},\ref{IASDublin}}, 
A.~M.~Glauser\inst{\ref{ETHZ}}, G.~Olofsson\inst{\ref{AlbaNova}}, %\\
%E.~Pantin\inst{\ref{SaclayCEAGif}}, 
T.~P.~Ray\inst{\ref{IASDublin}}, S.~Scheithauer\inst{\ref{MPIA}}, B.~Vandenbussche\inst{\ref{KULeuven}}, L.~B.~F.~M.~Waters\inst{\ref{Radboud},\ref{SRON}},
%%%%%%%%%%%%%%%%%%%%%%%%%%%%%%%%%%%%%%%%%%
% (d) MINDS affiliates (PhDs, postdocs, ...~proposed by GTO members) alphabetic
%%%%%%%%%%%%%%%%%%%%%%%%%%%%%%%%%%%%%%%%%%
A.~M.~Arabhavi\inst{\ref{Kapteyn}}, %D.~Gasman\inst{\ref{KULeuven}},
%R.~Guadarrama\inst{\ref{AstroWien}},
S.~L.~Grant\inst{\ref{MPE}}, H.~Jang\inst{\ref{Radboud}}, J.~Kanwar\inst{\ref{Kapteyn},\ref{IWFGraz},\ref{TUGraz}}, 
%M.~Morales-Calder\'on\inst{\ref{CABMadrid}}, N.~Pawellek\inst{\ref{AstroWien},\ref{Konkoly}}, D.~Rodgers-Lee\inst{\ref{IASDublin}}, 
J.~Schreiber\inst{\ref{MPIA}}, K.~Schwarz\inst{\ref{MPIA}}, M.~Temmink\inst{\ref{Leiden}}, %M.~Vlasblom\inst{\ref{Leiden}},
%%%%%%%%%%%%%%%%%%%%%%%%%%%%%%%%%%%%%
% (e) MIRI Co-PIs that are NOT on MINDS, alphabetical: (note that Kay is no longer co-PI}
%%%%%%%%%%%%%%%%%%%%%%%%%%%%%%%%%%%%%
%L.~Colina\inst{12}, %T.~R.~Greve\inst{21}, 
G.~\"Ostlin\inst{\ref{AlbaNova}}%, G.~Wright\inst{\ref{AstroEdinburg}}
}
    %
    \institute{
    Institute of Astronomy, KU Leuven, Celestijnenlaan 200D, Leuven, Belgium \\
    \email{valentin.christiaens@kuleuven.be}
    \label{KULeuven}
    \and
    STAR Institute, Universit\'e de Li\`ege, All\'ee du Six Ao\^ut 19c, 4000 Li\`ege, Belgium
    \label{STARLiege}
    \and
    Max-Planck-Institut f\"{u}r Astronomie (MPIA), K\"{o}nigstuhl 17, 69117 Heidelberg, Germany
    \label{MPIA}
    \and
    Kapteyn Astronomical Institute, Rijksuniversiteit Groningen, Postbus 800, 9700AV Groningen, The Netherlands
    \label{Kapteyn}
    \and
    LESIA, Observatoire de Paris, Universit\'e PSL, CNRS, Sorbonne Universit\'e, Universit\'e de Paris, 5 place Jules Janssen, 92195 Meudon, France
    \label{LESIA}
    \and
    Universit\'e Paris-Saclay, CNRS, Institut d'Astrophysique Spatiale, 91405, Orsay, France
    \label{SaclayCNRSOrsay}
    % \and
    % Kavli Institute for Astronomy and Astrophysics, Peking University, Beijing 100084, People's Republic of China
    % \label{Kavli}
% --------------- for G.-D. Marleau:
    \and
    Fakult\"at f\"ur Physik, Universit\"at Duisburg-Essen, Lotharstra\ss{}e 1, 47057 Duisburg, Germany
    \label{Duisburg}
    %% commented out to help save lines in the author list
    %%   (exponents take up space!)
    \and
    Institut f\"ur Astronomie und Astrophysik, Universit\"at T\"ubingen, Auf der Morgenstelle 10, 72076 T\"ubingen, Germany
    \label{Tuebingen}
    \and
    Physikalisches Institut, Universit\"{a}t Bern, Gesellschaftsstr.~6, 3012 Bern, Switzerland
    \label{Bern}
% ---------------
    \and
    Leiden Sterrewacht, Leiden University, 2300 RA Leiden, The Netherlands
    \label{Leiden}
    \and
    Max-Planck Institut f\"{u}r Extraterrestrische Physik (MPE), Giessenbachstr.~1, 85748, Garching, Germany
    \label{MPE}
    \and
    Institut f\"ur Astrophysik, Universit\"at Wien, T\"urkenschanzstr.~17, A-1180 Vienna, Austria
    \label{AstroWien}
    \and
    ETH Z\"urich, Institute for Particle Physics and Astrophysics, Wolfgang-Pauli-Str.~27, 8093 Z\"urich, Switzerland
    \label{ETHZ}
    \and
    Universit\'e Paris-Saclay, Universit\'e Paris Cit\'e, CEA, CNRS, AIM, F-91191 Gif-sur-Yvette, France
    \label{SaclayCEAGif}
    \and
    Centro de Astrobiolog\'ia (CAB), CSIC-INTA, ESAC Campus, Camino Bajo del Castillo s/n, 28692 Villanueva de la Ca\~nada, Madrid, Spain
    \label{CABMadrid}
    \and
    INAF – Osservatorio Astronomico di Capodimonte, Salita Moiariello 16, 80131 Napoli, Italy
    \label{INAFNapoli}
    \and
    Dublin Institute for Advanced Studies, 31 Fitzwilliam Place, D02 XF86 Dublin, Ireland
    \label{IASDublin}
    \and
    Institutionen f\"or Astronomi, Stockholms Universitet, AlbaNova Universitetscentrum, 10691 Stockholm, Sweden
    \label{AlbaNova}
    \and
    Department of Astrophysics/IMAPP, Radboud University, PO Box 9010, 6500 GL Nijmegen, The Netherlands
    \label{Radboud}
    \and
    SRON Netherlands Institute for Space Research, Niels Bohrweg 4, NL-2333 CA Leiden, The Netherlands
    \label{SRON}
    \and
    Institut f\"ur Weltraumforschung, Austrian Academy of Sciences, Schmiedlstr.~6, A-8042, Graz, Austria
    \label{IWFGraz}
    \and
    TU Graz, Fakult\"at für Mathematik, Physik und Geod\"asie, Petersgasse~16 8010 Graz, Austria
    \label{TUGraz}
    % \and
    % Konkoly Observatory, Research Centre for Astronomy and Earth Sciences, E\"otv\"os Lor\'and Research Network (ELKH), Konkoly-Thege Mikl\'os \'ut 15-17, H-1121 Budapest, Hungary
    % \label{Konkoly}
    %\and
    % DTU Space, Technical University of Denmark, Building 328, Elektrovej, 2800 Kgs.~Lyngby, Denmark
    % \and
    % Department of Astronomy, Oskar Klein Centre, Stockholm University, SE-106 91 Stockholm, Sweden
    % \label{OskarSU}
    % \and
    %UK Astronomy Technology Centre, Royal Observatory Edinburgh, Blackford Hill, Edinburgh EH9 3HJ, UK
    %\label{AstroEdinburg}
    }
    %\\ \email{valentin.christiaens@uliege.be}
    %
    
    \date{Received 22 December 2023, Accepted 6 March 2024}
    
    \abstract
    % context heading (optional)  
    {Two protoplanets have recently been discovered within the PDS~70 protoplanetary \new{disk}. % disk using ground-based facilities.
    JWST/NIRCam offers a unique opportunity to characterize them and their birth environment at wavelengths that are difficult to access from the ground.}
    % aims heading (mandatory)
    {We image the circumstellar environment of PDS~70 %measure the NIR photometry of protoplanets PDS~70~$b$ and c
    at 1.87\,$\upmu$m and 4.83\,$\upmu$m, assess the presence of Pa-$\alpha$ emission due to accretion onto the protoplanets, and probe any IR excess indicative of heated circumplanetary material.}
    % methods heading (mandatory)
    {We obtained noncoronagraphic JWST/NIRCam images of PDS~70 within the MIRI mid-INfrared Disk Survey (MINDS) program. We leveraged the Vortex Image Processing (VIP) package for data reduction, and we developed dedicated routines for optimal stellar point spread function subtraction, unbiased imaging of the disk, and protoplanet flux measurement in this type of dataset. %, and made them available in VIP.
    A radiative transfer model of the disk was used to separate the % flux 
    contributions from the disk and the protoplanets.}
    % results heading (mandatory)
    {We redetect both protoplanets
    %in our %NIRCam
    %images, 
    and identify extended emission after subtracting a %optimal %radiative transfer 
    disk model, \vc{including} a large-scale spiral-like feature. We interpret its signal in the direct vicinity of planet $c$ as tracing the accretion stream that feeds its circumplanetary disk, while the outer part of the feature may rather reflect asymmetric illumination %of the edge 
    of the outer disk. %not captured in our disk model.
    We also report a bright signal that is consistent with a previously proposed protoplanet candidate enshrouded in dust \vc{near} the 1:2:4 mean-motion resonance with planets $b$ and $c$. %is also recovered near its expected location at the date of our observation. %hence discuss the possibility of either a third protoplanet enshrouded in dust or a dust clump on a 
    %1:2:4 mean-motion resonance with planets $b$ and $c$.
    % prev version:
    % The 1.87\,$\upmu$m flux of planets $b$ and $c$ is consistent with atmospheric model predictions, which sets 3$\sigma$ 
    % upper limits on Pa-$\alpha$ line emission in line with previous accretion rate estimates, 
    % the above is now superseded with:
    The 1.87\,$\upmu$m flux of planet $b$ is consistent with atmospheric model predictions, %from the literature, 
    but the flux of planet $c$ is not. We discuss potential origins for this discrepancy, including %the possibility of 
    significant Pa-$\alpha$ line emission.     
    %of $1.44 \times 10^{-18}$\,W\,m$^{-2}$ %$\vc{update} 
    %and $1.10 \times 10^{-18}$\,W\,m$^{-2}$ for $b$ and c, respectively. 
    %For both protoplanets, 
    The 4.83\,$\upmu$m fluxes of planets $b$ and $c$ %the protoplanets 
    suggest %the presence of 
    enshrouding dust or heated CO emission from their circumplanetary environment. %, and in the case of planet $b$ agree with prior VLT/NACO M-band measurements.% In particular, our new measurement for planet $b$ confirms the tentative excess with respect to photospheric models already suggested by a VLT/NACO M-band observation.
    }
    % conclusions heading (optional)
   {The use of image-processing methods that are optimized for extended disk signals on high-sensitivity and high-stability from %the James Webb Space Telescope (JWST) 
   \vc{JWST}
   can uniquely identify signatures of planet--disk interactions and enable accurate photometry of protoplanets at wavelengths that are difficult to probe from the ground. Our results indicate that more protoplanets can be identified and characterized in other JWST datasets.}
   
\keywords{protoplanetary disks -- planet-disk interactions -- stars: individual: PDS~70 -- techniques: image processing}
    \maketitle
    
\defcitealias{Mesa2019a}{M19}

\section{Introduction}
    The direct detection and characterization of protoplanets is an emerging research field that is enabled by the latest generations of instruments \vc{achieving} high contrast and high angular resolution that are available on the largest ground-based facilities \citep[e.g.,][]{Keppler2018, Hammond2023}. %(e.g., \citealt{Currie2023} and references therein). 
    While the application of post-processing algorithms designed for point-source detection to bright protoplanetary disks has led to a number of unconfirmed protoplanet candidates, the recent development of both iterative and inverse-problem image-processing approaches (IIPAs) is opening the way to unbiased high-contrast IR imaging of the birth environment %the birth cradle
    of planets \citep[e.g.,][]{Pairet2021,Flasseur2021,
    Juillard2023}. %The James Webb Space Telescope (JWST) 
    \vc{JWST} now offers the opportunity to use these optimized tools to deepen the search and characterization of protoplanets at wavelengths that are inaccessible or are difficult to observe from the ground. This Letter demonstrates the combined potential of JWST and IIPAs and % in this area, 
    uses %data of %the protoplanets of applied to 
    the PDS~70 %protoplanetary 
    system as a testbed.

   %[Paragraph on the star PDS 70]
    \object{PDS 70} is a young ($5.4\pm1$~Myr) K7IV star located at a distance of $113.4\pm0.5~$pc \citep{Muller2018,Gaia-Collaboration2021}. It is surrounded by a protoplanetary disk composed of a %dust-depleted, yet 
    water-rich inner disk \citep{%Hashimoto2012, 
    Dong2012, Perotti2023} that is separated from the outer disk by an annular gap extending up to a radius of 
    $\sim$54~au \citep{Long2018,Keppler2019}. %The inclination and position angle of the semi-major axis of the disk are $49\fdg7\pm 0\fdg3$ and $158\fdg6\pm 0\fdg7$, respectively \citep{Keppler2018}.
   %[Paragraph summarizing previous results on the protoplanets - including past spectral analyses]
   Two nascent planets were imaged and confirmed independently in this large %annular 
   gap, at multiple near-IR (NIR) and submillimeter (submm) wavelengths, as well as in the H$\alpha$ line filter \citep[e.g.,][]{Keppler2018,Muller2018,Christiaens2019, Haffert2019, Isella2019, Benisty2021}. %, Casassus2022}.
   A third protoplanet candidate was also reported %possible 
   %protoplanet candidate 
   on a potential $\sim$13.5\,au orbit %based on SPHERE/IFS data 
   \citep[][hereafter \citetalias{Mesa2019a}]{Mesa2019a}.
   The system is therefore an ideal laboratory in which to study planet-disk interactions and search for accretion signatures. Hydrodynamical simulations suggest that the large gap is dynamically carved by both planets, and the (near) 2:1 mean-motion resonance observed by \vc{GRAVITY} \citep{Wang2021} %between the planets 
   %potentially
   is explained as the outcome of planet migration followed by resonance capture \citep{Bae2019, Toci2020}. %, Wang2021}. 
   Constraints on the distribution and depletion %, and in particular the depletion, 
   of dust and gas in the disk were recently inferred through radiative transfer and thermo-chemical \vc{forward modeling} of the NIR and submm  observations \citep{Portilla-Revelo2022,Portilla-Revelo2023}. 
   Submm continuum and NIR observations also 
   identified an arm-like structure that is hypothesized to either trace an asymmetric outer ring or a gap-induced 
   vortex \citep[][]{Isella2019, %Benisty2021, 
   Juillard2022}. %A point-like feature was also identified as a %possible 
   %protoplanet candidate on a $\sim$13.5\,au orbit %based on SPHERE/IFS data 
   %\citep[][hereafter \citetalias{Mesa2019a}]{Mesa2019a}.
    
   %Recent efforts have focused on connecting the disk chemical inventory to the atmospheric composition of the forming giant planets \citep{Facchini2021,Cridland2023}, but require better spectroscopic constraints (e.g.,~on the C/O ratio of the planets) to better constrain formation theories. 
   The spectral characterization of protoplanets $b$ and $c$ favors dust-enshrouded atmospheres combined with potential IR excess \citep{Muller2018, Christiaens2019a, %Mesa2019a, 
   Wang2020, Stolker2020, Wang2021}. % although %proximity of the bright edge of the outer disk has complicated photometric extractions for planet $c$ in most datasets 
   The cross-correlation of either these best-fit models or molecular templates with medium-resolution spectra did not detect the planets, however
   \citep{Cugno2021}. This suggests that either significantly more extinction affects the two protoplanets than fitting of the spectral energy distribution (SED) implies or that the self-consistent atmospheric models %commonly
   that have been used to characterize adolescent substellar objects do not describe embedded protoplanets. Hybrid models with contributions from surface accretion shocks \citep[e.g.,][]{Aoyama2020, Aoyama2021}, 
   or circumplanetary disk models may more accurately account for the measured SED \citep[e.g.,][]{%Szulagyi2019, 
   Chen2022}. %, Portilla-Revelo2022}. 
   %Alternatively, significantly larger extinction levels  than derived from SED fitting affect both protoplanets.
   In this context, observations at additional wavelengths are key to distinguish the different models. %Detection of diffuse and unresolved sub-mm continuum emission around planets $b$ and c, respectively, suggests the presence of circumplanetary dust \citep{Isella2019, Benisty2021}. For planet c, an upper mass limit of 0.7 $M_{\oplus}$ was estimated for its circumplanetary disk assuming the signal is optically thick \citep{Portilla-Revelo2022}. 

   %Recent observations of the system using the Mid-Resolution Spectrograph of the Mid-InfraRed Instrument of JWST witnessed the presence of water vapour in the inner disk of PDS~70, as well as significant variability of the MIR spectrum \citep{Perotti2023}. This implies that the innermost regions of the disk are dynamically perturbed leading to changing amounts of stellar illumination received in different parts of the disk. \gp{At the same time, NIR observations using VLT/XShooter confirmed the existence of magnetically inner disk wind in the system \citep{Campbell-White2023}, whose existence was previously suggested by \citet{Thanathibodee2020}.}
   
% In this Letter, we focus on JWST/NIRCam observations of PDS~70. % at 1.874~$\upmu$m and 4.834~$\upmu$m. 
% We report the first protoplanet images obtained with JWST, provide additional measurements to the SED of the two protoplanets, and summarize our search for infrared circumplanetary disk signatures around each of them.
%In Sect.~\ref{sec:obs} we describe our observations and their calibration. Methods used to process the calibrated images are presented in Sect.~\ref{sec:methods}. In Sect.~\ref{sec:results}, we then detail the results obtained from our image processing. In Sect.~\ref{sec:discussion}, we discuss the implications of our measured photometry for the protoplanets, and summarize the conclusions of our work.

\begin{table} 
\begin{center}
\caption{Properties of the star(+disk), protoplanets $b$ and c, and candidate $d$ inferred from the NIRCam F187N and F480M observations.} 
\label{tab:results}
\begin{tabular}{lcc}
\hline
\hline
Parameter & F187N & F480M \\
\hline
\multicolumn{3}{c}{PDS~70}\\
\hline
%Flux (mJy) & $291.6 \pm 2.9$ & $165.0 \pm 1.7$ \\ % v2
Flux (mJy) & $310.0 \pm 3.1$ & $160.9 \pm 1.6$ \\
\hline
\multicolumn{3}{c}{PDS~70~$b$}\\
\hline
Sep.\ (mas)$^{\rm (a)}$ & $164.5 \pm 10.4$ & $145.6 \pm 20.4$ \\
PA ($\degr$)$^{\rm (b)}$ & $129.0 \pm 4.1$ & $130.4 \pm 6.7$ \\
%Contrast$^{\rm (c)}$ & $(3.28 \pm 0.78) \times 10^{-4}$ & $(3.93 \pm 1.11) \times 10^{-3}$\\ %roll
Contrast$^{\rm (c)}$ & $(2.89 \pm 0.78) \times 10^{-4}$ & $(3.44 \pm 1.50) \times 10^{-3}$\\ %mean
%Flux ($\upmu$Jy) & $100.7 \pm 23.9$ & $ 682.6 \pm 192.8$ \\ % v2 for stellar flux
%Flux ($\upmu$Jy) & $101.7 \pm 24.2$ & $ 632.3 \pm 178.6$ \\ % roll
Flux ($\upmu$Jy) & $89.6 \pm 24.2$ & $ 553.5 \pm 
241.4$\\
%178.6$ \\ % mean
% Flux (SI) & $(6.93 \pm 1.66) \times 10^{-17}$ & $(7.56 \pm 2.14) \times 10^{-17}$ \\
%Flux$^{\rm (e)}$ & $(6.93 \pm 1.66) \times 10^{-17}$ & $(7.56 \pm 2.14) \times 10^{-17}$ \\
\hline
\multicolumn{3}{c}{PDS~70~$c$}\\
\hline
% BELOW ARE THE ROLL+NEGFC results => commented
% Separation$^{\rm (b)}$ & $216.9 \pm 21.5 $ & $216.7 \pm 20.6$\\
% PA$^{\rm (c)}$ & $267.2 \pm 7.0$ & $272.8 \pm 3.6$ \\  %7.0 instead of 3.5
% Contrast$^{\rm (d)}$ & $(8.82 \pm 6.36) \times 10^{-5}$ & $(1.55 \pm 0.30) \times 10^{-3}$\\
% Flux ($\upmu$Jy) & $(1.87 \pm 1.34) \times 10^{-17}$ & $(2.98 \pm 0.57) \times 10^{-17}$\\
%Flux$^{\rm (e)}$ & $(1.87 \pm 1.34) \times 10^{-17}$ & $(2.98 \pm 0.57) \times 10^{-17}$\\
% BELOW ARE THE IPCA results
Sep.\ (mas)$^{\rm (a)}$ & $195.7 \pm 23.3 $ & $216.7 \pm 20.6$\\
PA ($\degr$)$^{\rm (b)}$ & $269.7 \pm 6.8$ & $272.8 \pm 3.6$ \\  %7.0 instead of 3.5
Contrast$^{\rm (c)}$ & $(1.37 \pm 0.42) %$(1.12 \pm 0.64) 
\times 10^{-4}$ & $(1.47 \pm 0.30) \times 10^{-3}$\\
%Flux ($\upmu$Jy) & $35.6 \pm 12.0$ & $ 269.2 \pm 52.1$\\ % v2 for stellar flux
Flux ($\upmu$Jy) & $42.5 \pm 13.0$ %$34.7 \pm 19.8$ 
& $ 236.5 \pm 48.3$\\
%Flux (partly SI) & $ 35.6 \pm 12.0 & $(2.98 \pm 0.57) \times 10^{-17}$\\
%Flux$^{\rm (e)}$ & $(1.87 \pm 1.34) \times 10^{-17}$ & $(2.98 \pm 0.57) \times 10^{-17}$\\
\hline
\multicolumn{3}{c}{PDS~70~$d$?}\\
\hline
Sep.\ (mas)$^{\rm (a)}$ & $103.4 \pm 23.2 $ & -- \\
PA ($\degr$)$^{\rm (b)}$ & $293.0 \pm 12.7$ & -- \\  %7.0 instead of 3.5
Contrast$^{\rm (c)}$ & $(1.88 \pm 0.69) \times 10^{-4}$ & -- \\
Flux ($\upmu$Jy) & $61.5 \pm 22.6$ & -- \\
%Flux$^{\rm (e)}$ & $(1.87 \pm 1.34) \times 10^{-17}$ & $(2.98 \pm 0.57) \times 10^{-17}$\\
\hline
\end{tabular}
\end{center}\addvspace{-0.7em}
{\small%
Notes: %$^{\rm (a)}$Spectral flux density in mJy.
$^{\rm (a)}$Radial separation. $^{\rm (b)}$Position angle measured east of north. $^{\rm (c)}$Contrast ratio with respect to the star and unresolved inner disk. %\vc{update fluxes with latest calibration}
%$^{\rm (e)}$Spectral flux density in Wm$^{-2} \upmu$m$^{-1}$.
}
\end{table}
    
\section{Observations and image processing} \label{sec:obs}
    \label{sec:image_proc}

       \begin{figure*}[!t]
    \centering
    % \includegraphics[width=\columnwidth]{Fig1b_NRCB1.pdf}
    % \includegraphics[width=\columnwidth]{Fig1c_NRCB1.pdf}\\
    % \includegraphics[width=\columnwidth]{Fig1e_NRCBLONG.pdf}
    % \includegraphics[width=\columnwidth]{Fig1f_NRCBLONG.pdf}
    \includegraphics[width=\textwidth]{Fig1_v10.pdf}
    \caption{%v1 NIRCam images of PDS 70 obtained at 1.87$\upmu$m (F187N; top row) and 4.80$\upmu$m (F480M; bottom row) after roll subtraction of the stellar PSF. The circles indicate the predicted location of the two protoplanets based on the orbital fits presented in \citet{Wang2021}. These match observed overdensities in the images. The inner 5x5 pixels are masked in the 1.87$\upmu$m image. Images are low-pass filtered with a 2D Gaussian kernel with FWHM set to half the observed PSF FWHM. The arrow indicates the location of excess emission towards the northwest of the disk.
    %v3
    NIRCam images of PDS 70 obtained in the F187N (top row) and F480M (bottom row) filters %after %roll subtraction (first column) and 
    using our iterative PCA algorithm. % (first column). %(second column). 
    The second and third columns show the images obtained after subtraction of our outer-disk model %the optimal radiative transfer model of the outer disk, %inferred by NEGFD, 
    and after further subtraction of protoplanets $b$ and c, respectively. The major and minor axes of the disk are indicated in the first column with solid and dashed lines, respectively. The dashed circles indicate the predicted locations of protoplanets $b$ and $c$ based on the orbital fits in \citet{Wang2021} and the location of candidate $d$ based on the orbit suggested in \citetalias{Mesa2019a}. The astrometric measurements for $d$ (blue dots) are compared to our new estimated astrometry (solid circle) in panel c. 
    %The grey cross indicates the location of the star for reference.
    %The thick arrow indicates the spiral accretion stream associated with planet $c$. 
    %These match observed overdensities in the images. 
    %A 1-FWHM diameter mask hides the central part of the 1.87$\upmu$m image.  %The dotted line indicates the disk semi-minor axis, and the arrow points to the location of excess emission towards the northwest of the disk. 
    %The double arrow indicates a tentative fork %suggesting two different components
    %Arcs indicate the accretion stream feeding $c$, potential variable illumination effects, and an arm-like feature, identified in prior literature. % in the outer wake of planet $c$.
    The units are MJy sr$^{-1}$. %\vc{Redo panels c and f with arcs labeled 'illumination?' and  'stream', removing the thick arrow there. Smaller d circle in panel f. Update $b$ sub. Test appendix: Sub c with Wang prediction. Update sub. b with final contrast (2.96e-3)}
    }
    \label{fig:FinalImages}
    \end{figure*}
    
% _____________________________________
% Summary of the observations table
% \begin{table*}[ht] 
% \begin{center}
% \caption{Summary of the {\sc JWST/NIRCam} observations of PDS~70 used in this work. \vc{To be adapted for NIRCam}} 
% \label{tab:results}
% \begin{tabular}{lcccccccccc}
% \hline
% \hline
% Date & Strategy & Program & Filter & Coronagraph  & 
% T$_{\rm int}^{\rm (a)}$ & number &seeing$^{\rm (c)}$ & $\Delta$PA$^{\rm (d)}$ \\
% & & & & &
% [sec] & of frames$^{\rm (b)}$ & [\arcsec] & [\degrr] \\ 
% \hline

% 2023-03-08 & ADI & \href{http://archive.eso.org/wdb/wdb/eso/sched_rep_arc/query?progid=095.C-0298(A)}{095.C-0298(A)}  & $H23$  & N\_ALC\_YJH\_S  &
% 64 & 16 & 1.25 & 36.0\\

% 2023-03-08 & ADI & \href{http://archive.eso.org/wdb/wdb/eso/sched_rep_arc/query?progid=095.C-0298(B)}{095.C-0298(B)}  & $H23$  & N\_ALC\_YJH\_S  &
% 64 & 16 & 1.33 & 40.0\\

% \hline
% \end{tabular}
% \end{center}
% Notes: $^{\rm (a)}$ Exposure time of OBJECT frames. $^{\rm (b)}$ Number of frames in cubes. $^{\rm (c)}$ Average seeing from \href{https://www.eso.org/asm/ui/publicLog}{ESO Astronomical Site Monitoring (ASM)}. $^{\rm (d)}$ Maximum parallactic angle variation.. 
% \end{table*}
% _____________________________________

    %\subsection{Observations}
    We observed PDS 70 with JWST/NIRCam \citep{Rieke2005} as part of the MIRI guaranteed time observations on protoplanetary disks in the MINDS survey (PI: Th.~Henning, PID: 1282) %; Henning et al. in prep.) %; Henning et al. in prep.) 
    on 8 March 2023. The images were acquired simultaneously in the F187N ($\lambda_{\rm pivot}=1.874$\,$\upmu$m, $\Delta \lambda=0.024$\,$\upmu$m) 
    and F480M ($\lambda_{\rm pivot}=4.834$\,$\upmu$m, $\Delta \lambda=0.303$\,$\upmu$m) 
    filters without the coronagraph, with pixel scales of 31 and 63 mas/pixel, respectively. %No coronagraph was used.
    %The observations were made with 
    We used the smallest subarray (SUB64P), with an effective integration time of 0.35112~s, seven groups per integration, 142 integrations per exposure, and five dithered exposures at two
    %different 
    roll angles each, separated by $\sim$5.0\degr. The total effective integration time was thus $\sim$8.3 min. %, with each filter. 
    %A summary of the observations is provided in Table~\ref{tab:results} ***to be adapted.
    
    %\section{Calibration and pre-processing} \label{sec:preproc}
    We calibrated %the official JWST pipeline for calibration of the data , more specifically the stage 1 and stage 2. We used the same default parameters for stage 1 as used by 
    %\texttt{spaceKLIP} \citep[v.0.1.dev364;][]{Kammerer2022}, a Python package developed for high-contrast imaging with JWST instruments, as a wrapper for 
    the data with stages 1 and 2 (Detector1 and Image2) of the JWST pipeline \citep[v1.10.2;][]{Bushouse2023}, using the calibration reference data system context \texttt{jwst\_1166.pmap}.
    %with parameters left to the default \texttt{spaceKLIP} values \vc{To be checked}.
    %The parameters for Detector1 were set to the default values used in \texttt{spaceKLIP} \citep[v.0.1.dev364;][]{Kammerer2022}, and all other parameters left to default values in the JWST pipeline.
    The calibration %process 
    yielded two sets of five dithered images, % with suffix `outlier\_i2d', 
    each set corresponding to a different roll angle.
    %These calibrated images are then further considered for image pre-processing.
    %\section{Image processing}
    %\subsection{Pre-processing}\label{sec:preproc} 
    % Image processing
    %Characterization of both protoplanets and the disk first requires the subtraction of the bright glare from the star. This involves careful preparation of the images, stellar PSF modeling and subtraction. For these purposes, 
    We then relied on routines from the Vortex Image Processing \citep[VIP;][]{GomezGonzalez2017,Christiaens2023} package
    %\footnote{\url{https://github.com/vortex-exoplanet/VIP}}
    for image preprocessing (details in Appendix~\ref{sec:preproc}), stellar PSF subtraction, and \vc{forward modeling} of circumstellar signals.
    %% preprocessing
    % We first identified remaining bad pixels in the %'outlier\_i2d'
    % calibrated images through sigma-clipping, %using the \texttt{cube\_fix\_badpix\_clump} function of VIP, 
    % and corrected them using Gaussian kernel interpolation. We corrected the imperfect background subtraction done in the pipeline, which led to negative values, by subtracting the residual background level estimated far from the star.
    % We then found the stellar centroid coordinates %using the \texttt{cube\_recenter\_dft\_upsampling} routine in VIP, which first %leverages the single-step discrete Fourier Transform algorithm presented in \citet{Guizar-Sicairos2008} to 
    % %find
    % by finding shifts that optimize image cross-correlation throughout the sequence, and then %fits
    % fitting a 2D Gaussian model to the mean image of the aligned cube to find the centroid coordinates. %Depending on the adopted post-processing algorithm (see next paragraph), PSF subtraction is either done after image recentering or directly on the dithered images. In the latter case, the residual images are recentered using the shifts found by the \texttt{cube\_recenter\_dft\_upsampling} routine before PSF subtraction.
    % Throughout all stages of image processing, all shift and rotation operations were performed in the Fourier plane, the default behavior in VIP. %, as this better preserves pixel intensities \citep[e.g.,][]{Larkin1997}. %Image rotations are operated with the VIP implementation of the algorithm presented in \citet{Larkin1997}, which consists of three consecutive shear operations in the Fourier plane.
    % %Using Fourier-based operations turned out crucial for operations on non-coronagraphic images where the dynamic range of pixel intensities in the photometrically calibrated images spans 6 orders of magnitude. Inaccurate preservation of flux can indeed lead to more mismatches between the PSF images obtained at two different roll angles, hence lower achieved contrast.
    As the star did not saturate \new{the detector}, %in the observations, 
    we directly performed aperture photometry on the preprocessed images. We set the largest aperture radius that fit within the SUB64P field of view. %, for a fraction of encircled energy near unity. %, and corrected our measurements for missing flux based on the expected encircled energy provided in the \texttt{webbpsf} package \citep[v1.1.1;][]{Perrin2014}.
    %The photometry of the star in
    Table~\ref{tab:results} reports the integrated photometry of the system. We conservatively 
    considered 1\% absolute flux calibration uncertainties. %, as per the JWST user documentation.

    % \begin{figure*}[!t]
    % \centering
    % % \includegraphics[width=0.33\textwidth]{Fig2a_NRCB1.pdf}
    % % \includegraphics[width=0.33\textwidth]{Fig2b_NRCB1.pdf}
    % %\includegraphics[width=\columnwidth]{Fig2c_NRCB1.pdf}\\
    % \includegraphics[width=\textwidth]{Fig2def_NRCBLONG.pdf}
    % % \includegraphics[width=0.33\textwidth]{Fig2e_NRCBLONG.pdf}
    % % \includegraphics[width=0.33\textwidth]{Fig2f_NRCBLONG.pdf}
    % \caption{Images obtained at 1.87$\upmu$m (F187N; top row) and 4.80$\upmu$m (F480M; bottom row) after subtraction of the stellar PSF using roll subtraction (first column), and additional subtraction of the best outer disk model using the negative fake disk technique (second column; details in Sec.~\ref{sec:NEGFD}). The third column shows the signal-to-noise ratio map after optimal disk subtraction. The circles indicate the predicted location of the two protoplanets based on the orbital fits presented in \citet{Wang2021}. \vc{to be adapted}}
    % \label{fig:NEGFD_and_NEGFC}
    % \end{figure*}
    
    %\subsection{PSF modeling and subtraction}\label{sec:PSFsub}

    We investigated different approaches for PSF modeling and subtraction (details in Appendix~\ref{sec:AltAlgos}). %, leveraging roll subtraction \citep{Schneider2003} or Reference star Differential Imaging \citep[RDI; e.g.,][]{Mawet2012}, with classical or iterative algorithms (details in Appendix~\ref{sec:AltAlgos}).
    %The algorithm that led to t
    The best-quality images of the system \new{were obtained with} our implementation of an iterative principal component analysis \citep[IPCA; e.g.,][]{Pairet2021, Stapper2022} that leverages roll angle diversity. Its principle relies on estimating the circumstellar signals in the processed image that are obtained at each iteration and removing them from the %pre-processed 
    images that are used to create a PSF model (i.e., for each image, those obtained at the other roll angle) in the next iteration.
    % Moved to results:
    %Fig.~\ref{fig:FinalImages}a and d shows our IPCA results for the F187N and F480M data. IPCA recovered faint circumstellar signals originally hidden in the PSF wings while iteratively correcting for geometric biases affecting extended signals \citep[see 
    %e.g.,%Fig.~1 in
    %][]{Juillard2022}. 
    \new{We applied IPCA to a test dataset to illustrate that it reliably recovers point-like and extended circumstellar signals (Appendix~\ref{sec:TestIPCA}).}
    %This can progressively correct geometric biases affecting extended signals \citep[e.g.,%Fig.~1 in
    %][]{Juillard2023}.
%\subsection{Negative fake disk technique (NEGFD)}\label{sec:NEGFD}

    %Describe optimal radiative transfer model of the disk (NEGFD approach). 
    Before the photometry of the protoplanets is extracted, the expected contribution of the disk needs to be removed. This is particularly relevant for planet c, which is located near the bright edge of the outer disk. %Our goal in this work is not a full modeling of the disk, as this would involve a combined SED and disk image fit \citep[e.g.,][]{Keppler2018}. Instead, w
    We relied on the latest radiative transfer models of the disk %to start with a good approximation of the outer disk, 
    %obtained with \texttt{MCMax3D} \citep{Min2009}, as 
    detailed in \citet{Portilla-Revelo2022, Portilla-Revelo2023}, %We only allow these models to vary to a small extent while searching for the optimal disk model, to ensure they would still be compatible with other observational constraints of the system. %, such as the SED and ALMA observations.
    %The procedure we followed, 
    and followed a procedure that we refer to as the negative fake disk technique (NEGFD; details in Appendix~\ref{sec:NEGFD}).
%\subsection{Negative fake companion technique (NEGFC)}\label{sec:NEGFC}
    After removing the disk contribution, we used the negative fake companion \citep[NEGFC; e.g.,][]{Lagrange2010} technique to extract the exact astrometry and photometry of the protoplanets in both datasets (details in Appendix~\ref{sec:NEGFC_optim}).
    %We considered two different approaches, either applying NEGFC combined with roll subtraction using a forward modeling approach, %(i.e., the classical method), 
    %or NEGFC directly applied to the IPCA images, which lead to consistent results. 
    The final astrometry and photometry we retrieved for the protoplanets %by NEGFC combined with roll subtraction 
    are presented in Table~\ref{tab:results}. %, with the choices made for, and results obtained with each approach detailed in 
    % We considered the results obtained by NEGFC combined with roll subtraction, except for planet $c$ in the F187N data, where we derived astrometry and photometry directly in the IPCA image, as the planet is too self-subtracted after roll subtraction (for more details on our NEGFC approach, %and a comparison of results with either approaches, 
    %Appendix~\ref{sec:NEGFC_optim} and Table~\ref{tab:comparison_NEGFC}, respectively.% and Table~\ref{tab:comparison_NEGFC}, respectively). %These are consistent with the values directly inferred in the IPCA images (Table~\ref{tab:comparison_NEGFC}). 

    % We incorporated our new routines for iterative roll subtraction, IPCA, NEGFD, and Hessian-determinant based NEGFC in VIP (available as of v1.6.0).
    % Markov-Chain Monte Carlo approach. We considered the uncertainties provided in the . For NIRCam, this encompasses different sources of uncertainty (flat, Poisson, fringe), and to first order we can assume it is normally distributed considering the convergence theorem. The log-likelihood used in the MCMC algorithm ({\sc emcee}, VIP).
    
    % In the case of planet c, we include an additional source of uncertainty corresponding to the unknown true disk flux at the location of planet $c$. This is estimated by constructing a grid of radiative transfer disk models exploring different grain size distributions, based on the literature. Namely, *** [check Keppler2018, Wang2020, Toci2021, Portilla-Revelo2022]
    % and considering the standard deviation of disk flux measured in an aperture of the same size and at the same location as used for the extraction of the flux of planet $c$. Our final uncertainty on the flux of planet $c$ is a quadratic combination of the uncertainties corresponding to stellar residual noise and disk flux uncertainty.
    
\section{Results and discussion}\label{sec:results}

\subsection{Spiral accretion stream or variable illumination?}\label{sec:spiral}

Panels a and e in Figure~\ref{fig:FinalImages} show the F187N and F480M images we obtained with IPCA, respectively. IPCA recovered faint circumstellar signals that were originally hidden in the wings of the stellar PSF while it iteratively corrected for geometric biases that affect extended signals in roll-subtracted images \citep[Fig.~\ref{fig:AltAlgos}a and e; 
    %see also %Fig.~1 in
    ][]{Juillard2022}. 
% Fig.~\ref{fig:FinalImages}a and d show our F187N and F480M NIRCam images, respectively, after roll subtraction.
% Significant self-subtraction affects circumstellar signals in the images due to the 5\,$\degr$ difference between the two roll angles. At a given physical separation from the star, self-subtraction is stronger near the semi-minor axis of the disk due to the closer projected separation. %The separation-dependent amount of self-subtraction causes geometric biases in the map of circumstellar signals \citep[see also Figs.~1 in][]{Juillard2022, Juillard2023}.
% This accounts for the observed low intensity in the extracted image along the semi-minor axis on the near side of the disk (at a position angle PA$_b$$\sim$249\,$\degr$), where one expects a maximum in total intensity based on the scattering phase function of sub-$\upmu$m size dust grains \citep[e.g.,][]{Milli2017}.
% %\citep[e.g.,][]{Murakawa2010}. --> commented as just considers the polarization phase function.
% As seen in Fig.~\ref{fig:FinalImages}b and e, IPCA can recover adequately the flux levels of outer disk signals affected by both self- and over-subtraction 
% \citep[%Appendix~\ref{sec:AltAlgos}; 
% see also][]{Juillard2023}.
The IPCA images %also 
reveal signals of the inner and outer disk, signals at the expected location of the protoplanets, and a shift in the maximum intensity of the outer disk north of the semiminor axis $\beta$ on the near side of the disk (located at a position angle east of north PA$_{\beta}$$\sim$$249\degr$). In the absence of disk asymmetry or protoplanets, the maximum in total intensity is otherwise expected to be located along PA$_{\beta}$, based on the scattering phase function of sub-$\upmu$m size dust grains \citep[e.g.,][]{Milli2017}.
%asymmetric intensity distribution between the northern and southern parts of the outer disk with respect to the semi-minor axis on the near side of the disk (at a position angle PA$_b$$\sim$249\,$\degr$) (with respect to PA$_b$), 
To %better 
highlight %both 
the protoplanets and any disk asymmetry, we subtracted the %radiative transfer 
disk model found with NEGFD from the IPCA images (Fig.~\ref{fig:FinalImages}c and f). This revealed residual extended signals in addition to protoplanets $b$ and c, the predicted locations of which are indicated for the epoch of the observations. These predictions are based on the orbital fits presented in \citet{Wang2021}, and they are available through the platform \texttt{whereistheplanet} %\footnote{\url{http://whereistheplanet.com}}
\citep{Wang2021b}. 
% MOVED to next subsection:
%We note diffuse emission at the location of the point-like feature claimed in \citet{Mesa2019a}, whose predicted location which is likely tracing the north-western tip of the inner disk, and contributes to enhanced signals in the vicinity of planet $c$.
We do not detect any significant counterpart for the proposed submm continuum signal at the L5 Lagrangian point associated with planet $b$ \citep{Balsalobre-Ruza2023}.
We note an outstanding extended spiral-like signal connected with the position of planet $c$, however. It is indicated with a thick arrow in Fig.~\ref{fig:FinalImages}b and e.
%both the F187N and F480M images. %While this feature is bright in the direct vicinity of the planet, it is unclear whether it wraps around the cavity up to $\sim$180$\degr$ from the position angle of planet c, or whether the faint signals seen on the far side of the disk result from an inaccurate model prediction for back scattering. Regardless, w
%We interpret this feature as the spiral-accretion stream associated with planet c, given its similarity to predictions from dedicated 3D hydro-dynamical simulations of the system \citep[][Fig.~\ref{fig:SpiralTrace}]{Toci2020}.

\new{The dynamical interaction between a protoplanet and the disk in which it is embedded has long been known to cause spiral density waves \citep[][]{ Goldreich1979,Ogilvie2002} and angular momentum transport \citep[][]{Lin1979, Rafikov2002}. In the vicinity of the planet, the latter is expected to lead to a spiral-shaped accretion stream \citep[][]{Lubow1999, Ayliffe2009}. The %spiral 
accretion stream associated with %planet 
PDS~70 $c$ was predicted in a dedicated 3D hydrodynamical simulation in \citet{Toci2020}. Figure~\ref{fig:SpiralTrace} compares it with the spiral-like signal we identified in our observations.}
%We compared the spiral-like signal with the accretion stream associated with planet $c$ predicted in dedicated %3D 
%hydro-dynamical simulations
%of the system \citep[][Fig.~\ref{fig:SpiralTrace}]{Toci2020}. 
Although the agreement is remarkable, we note that the accretion stream mostly delimits the edge of the cavity. %Nonetheless, we also highlight that 
An asymmetric %variations in the 
illumination of the outer-disk edge that is not captured by our radiative transfer disk model could equally lead to an excess signal in our images after disk model subtraction. %our radiative transfer disk model.
%Varying illumination patterns cast from inner material onto the outer disk has been observed in the case of HD~135344~B \citep[][]{Stolker2017}, and evidence for a similar phenomenon also affecting PDS 70 disk images can be found in 
Multi-epoch polarized intensity images of the system show varying illumination and shadowing effects %patterns, affecting alternatively 
in the northwest and southeast parts of the outer disk \citep[e.g., Fig.~A2 in][]{Juillard2022}, suggesting that this is likely the dominant cause for the observed excess. % Given this observed variability, we expect this effect to be stronger than the signal 
This interpretation is consistent with the significant mid-IR (MIR) SED variability measured for the system, \vc{which} also suggests variable shadowing effects from the inner parts onto the outer parts of the disk \citep{Perotti2023}.

\begin{figure*}[!t]
\centering
\includegraphics[width=\textwidth]{Fig2_v9.pdf}
\caption{Composite SED of PDS 70 $b$ showing spectro- and photometric measurements from the literature (gray and black error bars, respectively), the new NIRCam F187N and F480M photometry %ic measurements 
(blue and red error bars, respectively), and %compared to 
the best-fit atmospheric models found in \cite{Wang2021}. The model with the most support is shown with a solid orange line (extinct BT-SETTL model with additional blackbody emission). It is consistent with both of our measurements, and suggests that circumplanetary contribution is required. %apart from atmospheric emission. 
The light blue error bar is obtained %by multiplying our measured contrast for the planet to 
considering photometry from the literature for the star. %instead of our NIRCam measurement, 
It illustrates the uncertainty associated with variability that affects \new{some} other data points of the SED (details in Appendix~\ref{sec:F187N_excess}). %from the star instead of our NIRCam images, shown for consistency with other works to calibrate the other data points of the SED.
}
\label{fig:spec_b}
\end{figure*}

While illumination effects most likely cause excess signals at the edge of the outer disk (upper arc in Fig.~\ref{fig:FinalImages}c), these effects alone %appear very unlikely 
%likely 
cannot account for the part of the spiral-like signal that is located inside the cavity based on the level of dust depletion therein %the cavity 
\citep[e.g.,][]{Dong2012, Keppler2018}. In the direct vicinity of planet $c$ (lower arc), a genuine dust density enhancement therefore appears to be required. Based on the clear visual connection to the location of protoplanet $c$, a spiral accretion stream feeding the circumplanetary disk of planet c that is detected in submm continuum observations \citep{Isella2019, Benisty2021, Casassus2022} is the most straightforward explanation for this signal.
%It is unclear whether part of the observed signal traces Pa-$\alpha$ emission from hot shocking gas \vc{Yuhiko/Gabriel: do you have a ref, or would suggest remove?}, which would require data from an adjacent continuum filter for confirmation.
Excess signal has tentatively been observed there in IIPA-processed VLT/SPHERE images  %inverse-problem based algorithms designed to better recover extended signals in high-contrast imaging sequences 
\citep[][]{Flasseur2021, Juillard2022}, although the nonremoval of %a radiative transfer 
an outer-disk model complicates an unambiguous identification of this signal in these images. 
Moreover, the location of the accretion stream is coincident with a %so-called 
gap-crossing spur found in ALMA CO observations of the disk \citep{Keppler2019}. This further strengthens the interpretation that this is an accretion stream. 
%Considering the detection, since then, of protoplanet $c$ \citep{Haffert2019}, its circumplanetary disk \citep{Isella2019, Benisty2021}, and now of an extended arc-like NIR signal in the direct vicinity of the protoplanet, the most likely explanation for all the observations is that both the spur and the arc-like NIR signal trace the accretion stream feeding the circumplanetary disk of $c$. 
This result suggests that some of the observed spiral features in less strongly inclined disks than PDS~70 could \new{also} be associated with embedded protoplanets \new{\citep[e.g.,][]{Dong2018, Ren2024}}. A similar spiral-shaped signal has recently been identified as connected to HD~169142~$b$ \citep{Hammond2023}. Likewise, a gap-crossing filament coincident with a twist in one of the %observed 
spirals of HD~135344~B may also \new{be caused by} an embedded protoplanet \citep{Casassus2021}.
%as part of the observed feature is located within the cavity, such as the inner wake with respect to the planet.

%It is likely that the same exercise of subtracting an optimal radiative transfer disk model to ground-based high-contrast images corrected from geometric biases (i.e., through an iterative or inverse-problem approach) would also have enabled the identification of this spiral accretion stream \citep[see e.g., the bulgy feature between the semi-minor axis and planet $c$ in Fig.~1 of][]{Juillard2022}. 
%In this same work, an outer arm-like feature was identified and characterized in VLT/SPHERE images. % without subtraction of a radiative disk model.
An arm-like feature was %also 
identified and characterized in the outer disk of PDS~70 using multi-epoch SPHERE images of the system obtained with IIPAs \citep{Pairet2021, Juillard2022}. The trace of this arm in a 2021 SPHERE dataset is compared with our IPCA image in Fig.~\ref{fig:SpiralTrace}c.
It is unclear whether it is associated with the spiral accretion stream %feeding the circumplanetary disk of planet c, 
or traces a separate feature, such as a vortex or an asymmetric second ring \citep[][]{Juillard2022}. %A tentative fork is seen $\sim$0$\farcs$5 to the north of the star in Fig.~\ref{fig:FinalImages}e/f (indicated with a double arrow). The bottom part of the fork 
The inner part of the arm at the edge of the cavity appears to be consistent with the %outer part of the 
part of the spiral-like signal seen in the F187N image, which may trace the varying illumination of %the edge of 
the outer-disk edge. The outer part of the arm appears to be consistent with the arm-like feature identified in previous SPHERE images. %\citep{Pairet2021, Juillard2022}. 
%a faint ring-like residual signal, as well as the lack of observed rotation over a 5-year baseline \citep{Juillard2022}, all argue in favour of one or both of the latter options.
% We suggest that the excess signal in the wake of planet $c$ seen in the F480M image is a combination of the spiral accretion stream seen in the F187N image and the outer arm feature reported in \citet{Juillard2022}, which at the angular resolution of the F480M image appears as a single broad arm feature. 
This %arm-like 
feature %in the outer wake of planet $c$ 
is the only signal %, only this arm-like feature is 
that is detected at a signal-to-noise ratio (S/N) $> 5$ in our F480M images in addition to the protoplanets (Fig.~\ref{fig:SNRmaps}) and was also observed in Keck/NIRC2 images of the system after a similar NEGFD procedure as we used \citep[Fig.~1 in][]{Wang2020}. 
%The absence of measured rotation motion over a 5-year baseline may rather point to another origin for the outer arm \citep[e.g., a vortex;][]{Juillard2022}, which appears associated to the sub-mm continuum excess identified at that location in \citet{Isella2019}.
%Alternatively the lack of measured rotation may suggest that the rigid-body rotation for spiral density waves may be inappropriate, at least far from the perturber body. 
%Carrying a similar analysis as in this work with ground-based archival data, and further monitoring of the system, may confirm both the spiral accretion stream and the origin of the outer arm.
% We suggest that the excess signal in the wake of planet $c$ seen in the F480M image is a combination of the spiral accretion stream seen in the F187N image and the outer arm feature reported in \citet{Juillard2022}.
%The F480M residual image also reveals another feature which lies outside the field of the F187N image: an 
% The arc-shaped positive residual signal $\gtrsim 0\farcs5$ to the south of the star may then also trace the continuity of the spiral accretion stream, whose intensity is modulated azimuthally by the scattering phase function (e.g., with a lower scattering efficiency on the far side of the disk). The prediction in \citet{Toci2020} shows indeed that the stream may be expected to wrap upstream of the planet and around the cavity over 360$\degr$.
%We hypothesize that this signal may trace the excess from the same spiral accretion stream as seen in the F187N image, but after a full revolution around the cavity, too faint to be visible on the far side of the disk due to the low efficiency of back scattering. 
% Alternatively, the arc may trace part of a second ring that is more visible along the semi-major axis, as hinted from the radial profile inferred from ALMA sub-mm continuum observations \citep{Benisty2021}.
An alternative origin for the arm-like signal is the presence of an as yet undetected planet in the outer disk that excites an inner spiral wake. We therefore investigated whether our data are sensitive to additional planets (Appendix~\ref{sec:contrast_curves}).
Our F480M contrast %curves 
and corresponding mass-sensitivity curves (Fig.~\ref{fig:contrast_mass_limits}) %, where we used both ATMO and BEX models for the conversion \citep{Phillips2020, Linder2019}. Our 5-$\sigma$ mass sensitivity curve 
constrain any %additional 
planets in the outer disk to have a mass below $\sim$1--2 $M_J$, neglecting extinction.
% A bright signal appears to stem out from the position of $c$. An outflow was suggested as the most likely interpretation for the detection of high-velocity CO for protoplanet HD~169142~$b$ \citep{Law2023}. Given the presence of a spiral accretion stream feeding the circumplanetary disk \citep{Benisty2021} and significant accretion rate \citep{Haffert2019}, a jet or outflow in the Pa-$\alpha$ line is not improbable. Alternatively, the signal may trace bright inner disk. Subtraction of continuum 

%We notice that the northwestern part of the disk shows $\sim$ 20\% excess intensity at both 1.87$\upmu$m and 4.8$\upmu$m, compared to its symmetric location with respect to the minor axis, in the south west of the image. In panel d, it can be seen that this part of the disk corresponds accordingly to a higher S/N. We also highlight the location of the arm-like feature characterized in \citet{Juillard2022} with an arrow. While at the angular resolution of the F480M image this feature is blended with the bright edge of the outer disk, it can be clearly seen after subtraction of the optimal disk model determined with NEGFD (Fig.~\ref{fig:NEGFD_and_NEGFC}e and f). %Depending on new mass-limits, keep or remove: As the new data do not permit better constraints to be set on this feature, we refer the interested reader to \citet{Juillard2022} for more ample discussion on the possible origins of this spiral-like feature.

% plain-text version in brackets to avoid "Token not alled in a PDF string" warnings
\subsection[Astrometry and photometry of protoplanets b and c]{Astrometry and photometry of protoplanets {\sf b} and {\sf c}}\label{sec:photometry}

The astrometry and contrast of planets $b$ and $c$ with respect to the star were inferred by using %directly the observed unsaturated PSF of the star in 
our NEGFC approach (Appendix~\ref{sec:NEGFC_optim}) after subtraction of our optimal disk model (Appendix~\ref{sec:NEGFD}). These contrast values were then multiplied by the integrated stellar flux values reported in Table~\ref{tab:results} to obtain the flux of the protoplanets. We note that the stellar fluxes in the F187N and F480M filters are $\sim$25\% and $\sim$7\% brighter than estimated based on the %calibrated
SpeX spectrum presented in \citet{Long2018} and the best-fit SED model to the extracted MIRI-MRS spectrum \citep[][%Jang et al. in prep.
]{Perotti2023}, respectively. %We found a $\sim$18\% excess at F187N and a , compared. Such observation appears compatible with 
These differences are % values are roughly 
compatible with the significant %inner disk 
IR variability of the star and inner-disk rim %in \citet{Casassus2022}, and confirmed at IR wavelengths 
\citep[e.g., $\sim$25\% variation at $\sim$5\,$\upmu$m between Spitzer and JWST observations;%brighter flux at 5\,$\upmu$m measured with Spitzer compared to JWST;
][]{Perotti2023}. %, %while the $\sim$25\% change may be accounted for considering %in addition that 
%large uncertainties affect ground-based photometry at 1.87\,$\upmu$m due to a strong water absorption band, and that the accretion process is variable, hence the 
%which is expected to also partly apply for 1.87 $\upmu$m flux measurements. %, where the contribution from the inner disk is still expected to represent $\sim$25\% of the total measured flux (compared to $\sim$60\% at 5\,$\upmu$m; Jang et al. in prep.) 
%Variability of the stellar Pa-$\alpha$ line, encompassed in the F187N filter, may also contribute to the variable 1.87\,$\upmu$m photometry. 
%All t
These considerations inspire caution regarding protoplanet photometry derived in contrast of the star \vc{that are} based on nonconcurrent absolute star photometry measurements. %(i.e., as is the case for most literature photometry). % (for multiplication to the measured protoplanet contrast). 
%We did not consider the absolute photometric calibration of the data themselves %as the photometric correction lead to inferred stellar photometry discrepant by up to two orders of magnitude with literature values, 
%to avoid unknown calibration errors associated with NIRCam photometry, %(J.~Girard, priv.~comm.) 
%and for consistency with past literature .

\begin{figure*}[!t]
\centering
\includegraphics[width=\textwidth]{Fig3_v7.pdf}
\caption{Same as Fig.~\ref{fig:spec_b}, but for PDS~70~$c$. %Composite SED of PDS 70 $c$ showing both literature spectro- and photometric measurements (grey and black error bars, respectively) and the new NIRCam F187N and F480M photometric measurements (blue and red error bars, respectively), compared to the best-fit atmospheric models found in \cite{Wang2021}. 
Here, the model with the most support is the plain Drift-Phoenix model \citep[details in][]{Wang2021}. %, shown with a green solid line. %The light blue error bar is obtained with a different absolute photometry for the star at 1.87$\upmu$m, used to obtain most other data points in this SED (see details in text).
}
\label{fig:spec_c}
\end{figure*}

%Table~\ref{tab:results} presents our extracted astrometry and photometry for the two protoplanets in the F187N and F480M images after subtraction of our optimal disk model. 
For each planet, the astrometric values we found for the two filters are consistent with each other. The F187N astrometry of planet %b is affected by the smallest uncertainty, while the larger uncertainty for 
$c$ and candidate $d$ is affected by large uncertainties owing to neighboring disk signals and the overlapping spiral accretion stream. For the F480M images, the large uncertainties also reflect the coarser angular resolution. All estimates are consistent with the expected astrometry of the two protoplanets at the date of our observations (given in the last column of Table~\ref{tab:comparison_NEGFC}), based on the orbital fits presented in \citet{Wang2021}. %$r_{b,{\rm exp.}} = 155.5 \pm 1.4$\,mas and PA$_{b,{\rm exp.}} = 132.6 \pm 0.3\degr$, and $r_{c,{\rm exp.}} = 210.1 \pm 1.0$\,mas and PA$_{c,{\rm exp.}} = 270.1 \pm 0.3\degr$.
%We note a slight offset in the direction of the outer wake, noticeable by comparing the centroid location with respect to the predicted location in the F187N and F480M. This may suggest some contribution from the accretion shock to the F480M flux.  
Considering our large uncertainties compared to ground-based measurements, we do not attempt new orbital fits %of the planets
in this work. %with these new measurements.

%\subsection{Photometry of protoplanets $b$ and c}

Figures~\ref{fig:spec_b} and \ref{fig:spec_c} show the %new 
SED of PDS 70 $b$ and $c$, respectively, including our NIRCam data at 1.87 and 4.83\,$\upmu$m. 
% PREVIOUS VERSION - not relevant for c anymore
% The F187N measurements for both protoplanets are consistent with predictions from the best-fit atmospheric models obtained with the BT-SETTL, DRIFT-PHOENIX and EXOREM grids %presented in 
% \cite[see][for details on the atmospheric models]{Wang2021}. %We refer the reader to \cite{Wang2021} for more details on the corresponding atmospheric models. 
% %We note however that large uncertainties affect the measurement for planet $c$. These reflect the fact that its flux is self-subtracted down to the noise level in the roll subtraction image, while bright extended signals surround its location in the IPCA image making an estimate of the contribution from the planet alone difficult. 
% \vc{update: not consistent for expected flux of $c$. 4 potential explanations: (i) potential star+inner disk variability (now confirmed); (ii) IFS spectrum extraction of $c$ biased by bright disk (creating a negative ADI side trace overlapping with $c$); (iii) best-fit models are therefore biased - in particular these could be much less extinct than inferred in \citet{Wang2021} (i.e., $A_V < 18$ mag); (iv) there may be significant Pa-$\alpha$ emission - the lack of image in an adjacent filter not including the line prevents a direct, model-independent, estimate of the line emission; (v) accretion variability for the protoplanet \citep[][]{Szulagyi2020, Casassus2022}.}
% Given the lack of measured excess at 1.87\,$\upmu$m, we estimate 3-$\sigma$ upper limits on Pa-$\alpha$ emission of $1.49 \times 10^{-18}$ W m$^{-2}$ and $1.22 \times 10^{-18}$ W m$^{-2}$, %(i.e., $\log(L_{{\rm Pa}\alpha, b}/L_{\odot}) < -6.24$ and $\log(L_{{\rm Pa}\alpha, c}/L_{\odot}) < -6.35$, respectively), 
% based on the difference between three times the uncertainties above our measurements and the best-fit atmospheric model for planets $b$ and $c$ (solid line in Figs.~\ref{fig:spec_b} and \ref{fig:spec_c}), 
% respectively. % and assuming all the excess flux would come from the emission line. 
% Assuming an extinction $A_V$$\sim$0.9 and $\sim$2.0 affecting planets $b$ and $c$ \citep{Uyama2021}, respectively, these limits correspond to Pa-$\alpha$ luminosity constraints $\log(L_{{\rm Pa}\alpha, b}/L_{\odot}) < -6.19$ and $\log(L_{{\rm Pa}\alpha, c}/L_{\odot}) < -6.24$, which convert to 3-$\sigma$ upper limits on the mass accretion rate of $\dot{M_b} < 1.0 \times 10^{-6} M_J$ yr$^{-1}$ and $ \dot{M_c}< 3.6 \times 10^{-7} M_J$ yr$^{-1}$ for planet masses of $M_b \sim 3.2 M_J$ and $M_c \sim 7.5 M_J$, respectively \citep{Wang2021}, considering the planet accretion models presented in \citet{Aoyama2020}.
% These constraints are in line with the accretion rates inferred from the measured H$\alpha$ flux for both planets \citep{Haffert2019}, considering the same model and assumptions: $\dot{M_b} \approx 3.2 \times 10^{-8} M_J$ yr$^{-1}$ and $ \dot{M_c} \approx 1.8 \times 10^{-8} M_J$ yr$^{-1}$. 
% %Our comparatively worse constraints mainly stem from the unknown contribution of nearby and overlapping extended signals to the measured F187N flux, captured in our large uncertainties propagated into mass accretion rate constraints.
% %The width of F187N is also larger than the resolution of  
% %Considering the presence of extended signals around planet $c$ in the IPCA image, we note that the most meaningful constraints on its Pa-$\alpha$ emission would require the subtraction of a neighbouring continuum channel, e.g., in a JWST/NIRSpec spectrum encompassing the line.
% NEW VERSION - not relevant for c anymore
The F187N measurement for planet $b$ is consistent with predictions from the best-fit atmospheric models obtained with the BT-Settl, Drift-Phoenix, and EXOREM grids %presented in 
%\citep[see][for details on the atmospheric models]
\citep[details in][]{Wang2021}. This does not hold for planet $c$. %We can assign 
This discrepancy can have various (not mutually exclusive) causes that we discuss in Appendix~\ref{sec:F187N_excess}. %Assuming other effects are insufficient to account for the excess, w
Here, we discuss 
the hypothesis of significant Pa-$\alpha$ line emission from the protoplanet that is not captured in the atmospheric model. % contributing to the total flux measured in the F187N filter. %We investigate this possibility below. % - the lack of image in an adjacent filter not including the line prevents a direct, model-independent, estimate of the line emission. 
% Assuming all the excess signal above the predicted atmospheric contribution comes from line emission, we estimate 
%% or (shorter):
If the excess signal above the model atmospheric contribution comes from Pa-$\alpha$ emission alone, the line flux is
$(5.7 \pm 2.7) \times 10^{-19}$ W m$^{-2}$.
Given the unknown amplitude of other biases discussed in Appendix~\ref{sec:F187N_excess}, this estimate is a conservative upper limit.
Considering $A_V \approx 2.0$ mag \citep{Uyama2021} and the distance of the system, our constraint on the Pa-$\alpha$ luminosity is  $\log(L_{{\rm Pa}\alpha, c}/L_{\odot}) \lesssim -6.5^{+0.2}_{-0.3}$, where $L_{\odot}$ is the solar luminosity. % considering the distance of the system.
\vc{When} a planet mass of $M_c \sim 7 M_J$ \citep{Wang2021} is assumed, the models in \citet{Aoyama2021} suggest that the mass accretion rate would be $ \log(\dot{M_c}/M_J$ yr$^{-1}) \lesssim -6.7^{+0.3}_{-0.2}$.
The same assumptions for the model and the planet applied to the H$\alpha$ flux reported in
\citet{Haffert2019} %one would expect %$\dot{M_c} \approx 1.8 \times 10^{-8} M_J$ yr$^{-1}$.
% This estimate is in line 
% with the accretion rates inferred from the measured H$\alpha$ flux for both planets 
% \citep{Haffert2019}, considering the same model and assumptions: %$\dot{M_b} \approx 3.2 \times 10^{-8} M_J$ yr$^{-1}$ and 
%$\dot{M_c} \approx 1.8 \times 10^{-8} M_J$ yr$^{-1}$
yield an accretion rate
$\log(\dot{M_c}/M_J$ yr$^{-1}) \approx -7.7$. If the \citet{Aoyama2021} models are an accurate representation of the accretion process onto giant planets, our results suggest that either additional sources of bias may have a non-negligible effect (e.g., stellar variability or underestimated SPHERE/IFS measurements; Appendix~\ref{sec:F187N_excess}), or
%above discrepancy suggests %may indeed, which is to be confirmed in 
%that %a re-analysis of 
%either the SPHERE/IFS measurements are indeed biased, requiring a re-analysis following a similar approach as in this work, or 
that protoplanets undergo significant accretion variability %, as is the case for stars, %in particular for hydrogen recombination lines 
\citep[e.g.][]{Szulagyi2020, Casassus2022}. %Our estimate on the accretion rate may then rather be considered in the meantime as an upper limit on the mass accretion, . 
%If the, an alternative way to solve this is by invoking protoplanet variability \citep[e.g.][]{Szulagyi2020, Casassus2022}, the amplitude of which is still to be constrained. 
With a similar reasoning as above, but using the 3 $\sigma$ uncertainty on the F187N flux measured for planet $b$ and an assumed extinction of $A_V \sim 0.9$ \citep{Uyama2021}, we constrain the Pa-$\alpha$ luminosity to $\log(L_{{\rm Pa}\alpha, b}/L_{\odot}) < -6.2$. %after considering the 
Based on the planet accretion models in 
\citet{Aoyama2021}, this translates into an %should correspond to an 
upper mass-accretion rate limit of %one considering a 3-$\sigma$ upper limit on the flux.3-$\sigma$ upper limits on the mass accretion rate of 
%$\dot{M_b} < 1.0 \times 10^{-6} M_J$ yr$^{-1}$ 
$\log(\dot{M_b}/M_J$ yr$^{-1}) < -6.0$
%and $ \dot{M_c}< 3.6 \times 10^{-7} M_J$ yr$^{-1}$ 
for a planet with a mass $M_b \sim 3 M_J$ %and $M_c \sim 7.5 M_J$, respectively 
\citep{Wang2021}. %, considering the planet accretion models presented in \citet{Aoyama2020}.
This constraint is in line with the accretion rates inferred from the H$\alpha$ flux measured for $b$ \citep{Haffert2019}, which for the same model and assumptions leads to %$\dot{M_b} \approx 3.2 \times 10^{-8} M_J$ yr$^{-1}$. 
$\log(\dot{M_b}/M_J$ yr$^{-1}) \approx -7.5$.
% and $ \dot{M_c} \approx 1.8 \times 10^{-8} M_J$ yr$^{-1}$. 
% %Our comparatively worse constraints mainly stem from the unknown contribution of nearby and overlapping extended signals to the measured F187N flux, captured in our large uncertainties propagated into mass accretion rate constraints.
% %The width of F187N is also larger than the resolution of  
% %Considering the presence of extended signals around planet $c$ in the IPCA image, we note that the most meaningful constraints on its Pa-$\alpha$ emission would require the subtraction of a neighbouring continuum channel, e.g., in a JWST/NIRSpec spectrum encompassing the line.
In summary, the measurement uncertainties and additional sources of bias together prevent us from confirming significant Pa-$\alpha$ line emission for the two protoplanets.

%\subsection{Potential excess at 4.8 $\upmu$m}
%**Infer upper limits on CO rovib emission or CPD thermal excess.
Our new F480M measurement for PDS~70~$b$ is consistent with the NACO $M$-band measurement presented in \citet{Stolker2020}. \citet{Wang2021} found that this point was driving the inclusion of an additional blackbody contribution that is representative of heated circumplanetary dust in the atmospheric model with the most support (solid line in Fig.~\ref{fig:spec_b}). 
%We caution that the coarse angular resolution makes an exact separation, which is driving the large uncertainty (Appendix~\ref{sec:NEGFC_optim} and Fig.~\ref{fig:NEGFC_safety_checks}).
%Our new photometric measurement lies $\gtrsim 2 \sigma$ higher than the best-fit atmospheric models that do not include any extra emission (DRIFT-PHOENIX and EXOREM).
For PDS~70~$c$, %the measured 
our 4.8 $\upmu$m photometry is roughly compatible with the best-fit Drift-Phoenix model, but it is significantly higher than the best-fit models from the other two grids, which are known to reproduce the spectra of old L-T dwarfs better \citep[e.g.,][]{Witte2011}.
For both planets, the tentative excess may be attributable to either a warm dusty environment or to ro-vibrational CO line emission from a heated circumplanetary disk \citep[e.g.,][]{Oberg2023}.
We defer a detailed spectroscopic analysis to a later study including NIRSpec measurements. The NIRSpec measurements %(R$\sim$2700 spectrum from 2.8 to 5.1 $\upmu$m)
have the highest potential to confirm the tentative 4.8 $\upmu$m excesses and constrain their origin.

\subsection{A third protoplanet, a dust clump, or an inner spiral?}

The brightest signals in our disk-subtracted F187N image (Fig.~\ref{fig:FinalImages}b) are found near the predicted location of a protoplanet candidate proposed in \citetalias{Mesa2019a} at the outer edge of the inner disk (referred to as a point-like feature therein). The candidate was found in SPHERE/IFS datasets acquired between May 2015 and April 2019. Its reported astrometry, indicated with blue dots in Fig.~\ref{fig:FinalImages}, is consistent with an $\sim$13.5\,au circular Keplerian orbit in a plane similar to that of the outer disk, which %. The latter 
is assumed for the predicted location shown with dashed circles in Fig.~\ref{fig:FinalImages}. %b and f. %\citep[$i_d \sim 46.5\degr$][]{Mesa2019a}.
For clarity, we also show the images that were obtained after the estimated flux of protoplanets $b$ and $c$ was also subtracted (Fig.~\ref{fig:FinalImages}c and f).
While inner-disk signals are present near the predicted location, there appears to be a significant excess compared to the disk signals alone, considering that 
the inner-disk emission originates in a symmetric location with respect to the minor axis %along the semi-major axis 
southeast of the star. 
% as a proxy for the expected 
%amount of 
%inner disk signals at the predicted location. % to be expected at a symmetric location to the north-west of the minor axis. %While the possibility of an artefact cannot be fully ruled out, the consistent photometry and astrometry measured for protoplanets $b$ and c, suggest that our IPCA image re down to a close separation from the star, uniquely enabled by the high-stability of the JWST/NIRCam point-spread function.
While a signal at a separation of $\sim$$2\lambda/D$ from the star should be considered with caution, the recovery of the inner disk with a geometry close to expected suggests that the PSF subtraction residuals are lower than the inner-disk signals. %leading us to consider 
This means that the observed excess is likely %being 
of 
circumstellar origin and not an % PSF subtraction 
artifact.
%Considering the above, we treat the signal as likely stemming from a compact astrophysical source, 
We therefore refer to it hereafter as ``$d$'' because it might trace a dusty feature or a third protoplanet. 

Table~\ref{tab:results} reports the astrometry and photometry we extracted for $d$ in the F187N image. While the F480M image also shows a bright pixel near the expected location, its separation from the star is too small %($\sim$1 pixel) to enable 
for a reliable contrast estimate. % to be made.
The contrast of $d$ derived in the F187N image is higher about a factor of 2 than the median contrast reported in $YJH$ bands in \citetalias{Mesa2019a}. Our estimate is affected by large uncertainties due to the unknown amount of contamination from the inner disk, and this excess is therefore only marginally significant. Nonetheless, we argue that this excess %difference in contrast 
is expected. The source spectrum is dominated by scattered stellar light at NIR wavelengths \citepalias{Mesa2019a}, which is consistent with a very dusty object. In this case, the scattering phase function of the total intensity is also expected to modulate its brightness along its orbit. Compared to prior epochs, the object is now closer to PA$_{\beta}$, and therefore, we would  % would 
expect a %significantly
brighter signal. We estimate an enhancement of a factor of $\sim$1.6 in reflected light for $d$ between April 2019 and March 2023 based on the variation in the flux of the outer disk measured at PA=PA$_{\beta}$-PA$_{d,2019}$ and PA=PA$_{\beta}$-PA$_{d,2023}$, where we considered the southwest part of the outer disk for this estimate to avoid any bias from the accretion stream toward the northwest. %(i.e., PA$_{\beta}$-PA$_d$ instead of PA$_d$-PA$_{\beta}$).
Within the uncertainties, our measured contrast is therefore consistent with tracing the same object as \citetalias{Mesa2019a}.
This %independent re-detection of $d$ at a location consistent with its expected position for the orbit proposed in \citetalias{Mesa2019a} 
new measurement adds 4 years %time baseline 
to the existing 4-year time baseline for the orbital coverage, and it significantly reduces the probability that $d$ either traces a moving illumination effect or the filtered northwestern tip of the inner disk (in \citetalias{Mesa2019a}). %While the resolution is lower in the NIRCam images, the signal appears to stick out from inner disk signals alone, now at a different PA than that of the major axis of the inner disk. 

% The high-angular resolution ALMA sub-mm continuum images reported in \citet{Casassus2022} suggest the presence of a clumpy signal to the northwest of the inner disk, at the rough location of , potentially made of a spiral component near the location of $d$. We tested whether this structure 
% The PA of $d$ is consistent with over a 8 years time baseline appears consistent, considering both the PA values reported in \citetalias{Mesa2019a} and our new measured PA. Considering the PA \vc{TBD: chi2 test for inner spiral hypothesis of b - is it ruled out?}

\new{Our independent redetection of a signal that is compatible with candidate $d$ does not unambiguously confirm its protoplanet nature, but}
%While the independent re-detection of $d$ with NIRCam does not confirm the protoplanet % candidate 
%hypothesis, %which would likely require a mid-IR detection for unambiguous confirmation, 
it raises the question of which other physical processes might give rise to a bright and compact NIR signal that moves at \vc{the} local Keplerian speed. %corresponding to its deprojected separation
%plausibly capture physically? 
The $YJH$ spectrum measured in \citetalias{Mesa2019a} is consistent with tracing scattered stellar light, and the authors therefore suggested that it might trace either a transient dust clump %, which requires their formation to be common, 
or a protoplanet enshrouded in dust. 
% It is unclear whether the former is a common occurrence in disks % if not asso to planet formation itself.
% %It is unclear whether the former is expected to be common, 
% while exoplanets are known to be ubiquituous  based on surveys of mature systems using indirect detection methods. %can be 
%expected to be common, but r
Radiative hydrodynamical simulations of embedded giant planets suggest that they become enshrouded in a dusty circumplanetary disk or envelope, which can display a scattered-light dominated spectrum at NIR wavelengths \citep[e.g.,][]{Szulagyi2019}.
%It is interesting to note that i
If $d$ indeed traces a protoplanet enshrouded in dust, its semimajor axis of $\sim$13.5\,au would place it near the 1:2:4 mean-motion resonance with planets $b$ and $c$ %, considering the latest orbital fits 
\citep[$a_b\approx 21$\,au and $a_c \approx 34$\,au;][]{Wang2021}. Follow-up studies of $d$
%particularly interesting 
are therefore especially exciting. % in order to confirm whether it traces indeed a protoplanet.
Distinguishing the hypotheses of a %transient
dust clump from the circumplanetary disk or an envelope will require MIR flux measurements \citep[e.g.,][]{Chen2022}. Because of the angular separation, this may need the advent of MIR imagers and spectrographs on extremely large telescopes \citep[e.g., ELT/METIS;][]{Brandl2018}. Figure~\ref{fig:schema} summarizes our proposed interpretation of the main features detected in our NIRCam observations of PDS~70.
%to achieve the necessary angular resolution at thermal IR wavelengths.
%Fig.**Appendix shows the image after the subtraction of our optimal candidate for 
 %This interpretation may also further account for the brightness asymmetry between north west and south east parts of the inner disk, as the excess signal could correspond to the outer spiral wake associated with d, a feature also tentatively seen in the F480M image,
 
%---------------------------------------------------------------

% \section{Summary and conclusions} % since we are aiming for a letter, this is merged with the conclusions
% \label{sec:conclusion}

% We identified an extended spiral-like feature in our F187N and F480M JWST/NIRCam images of PDS~70 after removing our best outer disk model. We interpret the part of the signal located in the direct vicinity of planet $c$ as tracing a spiral accretion stream feeding its circumplanetary disk, while the outer part of the feature maybe explained by a more illuminated edge of the outer disk than captured in our radiative transfer disk model. %, as suggested by variations between multi-epoch polarized intensity images of the disk \citep{Juillard2022}.
% The accretion stream was only tentatively identified in previous works, as clear identification required both the use of %novel techniques to correct for self-subtraction affecting classical PSF subtraction algorithms leveraging angular diversity, 
% IIPAs and subtraction of an appropriate outer disk model. Our result suggests that some of the observed spiral features in less inclined disks than PDS~70 may also be associated with embedded protoplanets. A similar spiral-shaped signal was recently identified as connected to HD~169142~$b$ \citep{Hammond2023}. Likewise, a gap-crossing filament coincident with a twist in one of the observed spiral arms of HD~135344~B may also correspond to an embedded protoplanet \citep{Casassus2021}.
% For both protoplanets, we also report tentative excesses at 4.8\,$\upmu$mcompared to expected atmospheric contributions alone, which requires further investigation with NIRSpec data %have the potential to confirm and identify the source of this potential excess, 
% for both confirmation and constraints on their potential circumplanetary origin. Finally, we also detect a bright signal at 1.87\,$\upmu$m consistent with the dust-enshrouded protoplanet candidate proposed in \citetalias{Mesa2019a}, considering the expected Keplerian motion for a planet on a $\sim$13.5~au orbit. Its radial location near 1:2:4 mean-motion resonance with planets $b$ and $c$ makes it an interesting candidate to follow-up in order to confirm whether it traces indeed a protoplanet.
%to verify whether they can be assigned to heated circumplanetary dust or CO emission.

%-----------------------------------------------------------------
\begin{acknowledgements}
We thank Jason Wang for sharing atmospheric models and GRAVITY spectra of the protoplanets. We also thank Yuhiko Aoyama, Faustine Cantalloube and Julien Girard for useful discussions. VC and OA thank the Belgian F.R.S.-FNRS, and the Belgian Federal Science Policy Office (BELSPO) for the provision of financial support in the framework of the PRODEX Programme of the European Space Agency (ESA) under contract number 4000142531. This project has received funding from the European Research Council (ERC) under the European Union's Horizon 2020 research and innovation programme (grant agreement No 819155), and from the Wallonia--Brussels Federation (grant for Concerted Research Actions).
G-DM acknowledges the support of the DFG priority program SPP 1992 ``Exploring the Diversity of Extrasolar Planets'' (MA~9185/1) and from the Swiss National Science Foundation under grant
200021\_204847
``PlanetsInTime''. Parts of this work have been carried out within the framework of the NCCR PlanetS supported by the Swiss National Science Foundation.
TPR acknowledges support from the ERC under grant 743029 (EASY).
This work is based on observations made with the NASA/ESA/CSA James Webb Space Telescope. The data were obtained from the Mikulski Archive for Space Telescopes at the Space Telescope Science Institute, which is operated by the Association of Universities for Research in Astronomy, Inc., under NASA contract NAS 5-03127 for JWST. %These observations are associated with program \#1282. %said in text
This work has made use of data from the European Space Agency (ESA) mission {\it Gaia} (\url{https://www.cosmos.esa.int/gaia}), processed by the {\it Gaia} Data Processing and Analysis Consortium (DPAC, \url{https://www.cosmos.esa.int/web/gaia/dpac/consortium}). Funding for the DPAC has been provided by national institutions, in particular the institutions participating in the {\it Gaia} Multilateral Agreement. 
%% Unclear whether to uncomment below (here it's NIRCam data on MIRI GTO time)
% The following National and International Funding Agencies funded and supported the MIRI development: NASA; ESA; Belgian Science Policy Office (BELSPO); Centre Nationale d’Etudes Spatiales (CNES); Danish National Space Centre; Deutsches Zentrum fur Luftund Raumfahrt (DLR); Enterprise Ireland; Ministerio De Econom\'ia y Competividad; Netherlands Research School for Astronomy (NOVA); Netherlands Organisation for Scientific Research (NWO); Science and Technology Facilities Council; Swiss Space Office; Swedish National Space Agency; and UK Space Agency.
This work benefited from the 2022 Exoplanet Summer Program in the Other
Worlds Laboratory (OWL) at the University of California, Santa Cruz, a program funded by the Heising-Simons Foundation.
\end{acknowledgements}

% to avoid spurious "natbib Warning: Citation multiply defined": use TeX Live 2020!
% Thanks to tech support from Overleaf, who pointed to:
% https://tex.stackexchange.com/questions/625901/overleaf-citation-multiply-defined-with-aa-class-file
\bibliographystyle{aa} % style aa.bst
\bibliography{references}

\clearpage

\begin{appendix}

\section{Image processing}

    \begin{figure*}[h]
    \centering
    \includegraphics[width=\textwidth]{FigA1_v4.pdf}
    \caption{Images obtained at 1.87$\upmu$m (F187N; top row) and 4.80$\upmu$m (F480M; bottom row) with mean roll subtraction, %PCA-ARDI,
    TRAP, IROLL, and IPCA. See text for the details of each algorithm. The images correspond to the largest common field of view probed by the dithering pattern employed during the observation. The plate scale is 31 and 63\,mas/pixel for the F187N and F480M images, respectively. A numerical mask with the radius set to the FWHM of the PSF covers the inner part of the F187N images. %is used \vc{TBD: replace PCA-ARDI with TRAP, and unmask images}
    }
    \label{fig:AltAlgos}
    \end{figure*}

\subsection{Preprocessing} \label{sec:preproc}

    %% preprocessing
    We first identified remaining bad pixels in the %'outlier\_i2d'
    calibrated images through sigma-clipping using the \texttt{cube\_fix\_badpix\_clump} function of VIP, 
    and we corrected for them using Gaussian kernel interpolation. We subsequently corrected for the imperfect background subtraction performed by the JWST pipeline that led to negative values by subtracting the residual background level estimated far from the star.
    We then found the stellar centroid coordinates using the \texttt{cube\_recenter\_dft\_upsampling} routine in VIP. This routine first leverages the single-step discrete Fourier transform algorithm presented in \citet{Guizar-Sicairos2008} to 
    find
    %by finding 
    shifts that optimize the image cross-correlation throughout the sequence, and it then fits
    %fitting 
    a 2D Gaussian model to the mean image of the aligned cube to find the centroid coordinates. %Depending on the adopted post-processing algorithm (see next paragraph), PSF subtraction is either done after image recentering or directly on the dithered images. In the latter case, the residual images are recentered using the shifts found by the \texttt{cube\_recenter\_dft\_upsampling} routine before PSF subtraction.
    Throughout all stages of the image processing, all shift and rotation operations were performed in the Fourier plane, the default behavior in VIP\footnote{\url{https://github.com/vortex-exoplanet/VIP}}, as this better preserves pixel intensities \citep[e.g.,][]{Larkin1997}. %Image rotations are operated with the VIP implementation of the algorithm presented in \citet{Larkin1997}, which consists of three consecutive shear operations in the Fourier plane.
    %Using Fourier-based operations turned out crucial for operations on non-coronagraphic images where the dynamic range of pixel intensities in the photometrically calibrated images spans 6 orders of magnitude. Inaccurate preservation of flux can indeed lead to more mismatches between the PSF images obtained at two different roll angles, hence lower achieved contrast.
    
\subsection{Subtraction of the point spread function}\label{sec:AltAlgos}
    
    We investigated different approaches for the modeling and subtraction of the PSF.
%%% Roll subtraction
    We first considered pair-wise roll-subtraction \citep[e.g.][]{Schneider2003} between individual dithered images acquired with each of the two roll angles. We also considered the pair-wise subtraction of the mean image of each of the two sets. We note a minor improvement in the residuals near the star that might be due to the spatial undersampling of the images \vc{which limits} the efficacy of individual pair-wise image subtractions. Both the mean and individual options are implemented in the \texttt{roll\_sub} function of VIP, which is available as of version 1.6.0. %The function allows the use of a threshold for the identification of signals to be propagated at each iteration. We recommend setting it manually to a value above the noise level of the image in the first leads to cleaner results. The exact value of this threshold, or leaving it to zero, does not significantly change our final images.
    The images were smoothed with a Gaussian kernel, the FWHM of which was set to 60\% of the observed width of the instrumental PSF. This helps to mitigate the undersampling that affects the F187N data and the pixel-to-pixel noise induced by the roll-subtraction approach. In Fig.~\ref{fig:AltAlgos}a and e, we show the images we obtained with mean roll subtraction on the F187N and F480M data, respectively. Low residuals are achieved owing to the stability of the observed PSF, but significant self-subtraction and geometric distortion of circumstellar signals can also be noted (e.g., by comparison with the IPCA images). Compared to the F480M images shown in Fig.~\ref{fig:FinalImages}, the F480M images in Fig.~\ref{fig:AltAlgos} correspond to the largest common field covered by all dither positions.

In addition to direct roll subtraction, we investigated alternative post-processing methods designed to work in combination with angular differential imaging, namely principal component analysis \citep[PCA;][]{Soummer2012, Amara2012} and temporal reference analysis for planets \citep[TRAP;][]{Samland2021}. We used the PCA algorithm implemented in VIP and used the images obtained at the other roll angle as input PCA library for each image. This is similar to smartPCA, which was proposed for angular differential imaging \citep{Absil2013}. This yielded very similar results to the roll-angle subtraction.
The TRAP algorithm computes a contrast map. A contrast was computed for each off-axis ($\Delta$RA, $\Delta$DEC) pair on-sky by simultaneously modeling the pixel light curves created by an off-axis PSF at the sky position and the temporal systematics of the pixels affected by this off-axis PSF. For our JWST data, there are measurements at two roll angles, such that the off-axis PSF model for each ($\Delta$RA, $\Delta$DEC) pair resulted in a step-function with varying jump heights depending on the location of each pixel relative to the off-axis PSF.
In practice, the contrast map resulting from the \vc{forward modeling} is a form of local deconvolution with a PSF model that takes the field-of-view rotation into account.
This approach is optimized for detecting point sources, but it also works out of the box. However, because the disk structure surrounding the host star is complex, it is difficult to locate uncontaminated reference pixels of the host star PSF to model the temporal systematics in the data. For our results, each off-axis PSF pixel light curve (step function) was fit simultaneously with only a constant factor. Because the instrument is stable, this provided good results. However, higher-order noise models or models informed by temporal trends seen in the guide-star observations may provide better results in the future. Especially when fitting a \vc{forward model} that explicitly models the entire scene including disks and planets, overfitting becomes less of a concern. The TRAP analysis is extremely efficient computationally. It requires less than 5 seconds on a laptop. The TRAP images obtained for the F187N and F480M data are shown in Fig.~\ref{fig:AltAlgos}b and f, respectively.

As a less aggressive alternative, we performed reference star differential imaging \citep[RDI; e.g.,][]{Mawet2012} using %either archival non-coronagraphic PSF observations or model PSFs created with the {\sc webbpsf} package \citep[v1.1.1;][]{Perrin2014} as reference PSF libraries. We leveraged RDI through 
%the principal component analysis \citep[PCA;][]{Soummer2012,Amara2012} 
the PCA algorithm implemented in VIP. 
We tested RDI using observed noncoronagraphic PSFs from the MAST archive as reference library, but this led to strong PSF residuals after subtraction.
It is unclear whether this is due to the field dependence of the PSF, undersampling effects, a different spectral slope for the source (for the F480M filter), or a combination of these factors.  
We therefore also tested creating reference stars using the {\sc webbpsf} package \citep[v1.1.1;][]{Perrin2014} as reference PSFs to test RDI. In the latter case, we considered the PSF distortion dependence on the location in the field. We built the synthetic PSFs considering a 3972~K blackbody \citep{Muller2018} as the stellar spectrum model, anticipating a potential impact of the spectral slope on the PSF for the F480M filter. We also used the optical path differences from the day following the observations, which appear identical to those measured 3 days before the observations. 
Upon carefully checking the radial profile of the {\sc webbpsf} synthetic PSFs and the observed PSF of PDS~70, we noted a broader core for the observed PSF of PDS~70 and residual diffuse hexagonal-shape stellar wings after subtraction. This may suggest a non-negligible contribution from marginally resolved inner-disk signals. We also tested PCA-RDI with a library of synthetic PSFs. These were produced separately for each observed dither position on the detector and considered a range of subpixel shifts around these dither locations (within 0.5px, with steps of 0.05px). Here, we attempted to emulate the broader core and undersampling effects. %, and provided this library as input to the PCA function in VIP (\texttt{pca} function) to perform PCA-RDI individually.\
This marginally improved the results, but still yielded strong PSF wing residuals, and hence, poor-quality final images. 

    %PSF subtraction with a library of synthetic PSFs encompassing only a $\sim$0.2px left strong residuals in the core. 
    %More than capturing the undersampling effect of the PSF, this test suggests marginally resolved inner disk emission. Upon carefully checking the radial profile of the {\sc webbpsf} synthetic PSFs and the observed PSF of PDS~70, we noticed a significantly broader PSF core for PDS~70 ***quantify from comparison to WebbPSF model*** and residual hexagonal-shape stellar wings. This may suggests a non-negligible contribution from marginally resolved inner disk signals. We therefore considered a library spanning wider shifts, up to 0.5px (consistent with the **\% larger PSF core). The resulting images are shown in Fig.~\ref{fig:FinalImages}b and e.
    %Derotation of residual images after model subtraction is carried out using the roll angles provided in the image headers. %Image rotations are operated with VIP's implementation of the fast Fourier transform algorithm presented in \citet{Larkin1997}, which consists of three consecutive shear operations in the Fourier plane.
    
As our RDI attempts were unfruitful, %(Appendix~\ref{sec:AltAlgos}), 
we instead focused on iterative approaches leveraging roll-angle diversity. 
We implemented an iterative roll (IROLL) subtraction algorithm similar to the one presented in \citet{Heap2000}, and an iterative principal component analysis algorithm \citep[IPCA; e.g.,][]{Pairet2021}. In either case, circumstellar signals estimated in the processed image obtained at each iteration (e.g., positive residuals above a certain threshold) were removed from the pre-processed images used to create a PSF model (i.e., for each image, those obtained at the other roll angle) in the subsequent iteration.
%For IPCA, we considered two different cases: a PCA library composed for each roll angle image of images obtained at the other roll angles (i.e., akin to IPCA-ADI with a rotation threshold), and a PCA library made of a combination of both alternate roll angle images and reference PSF images (i.e., akin to IPCA-ARDI with a rotation threshold). Both approaches yielded very similar results. For the IPCA-ARDI case, we considered reference images from the ***program***, and considered the same number of reference and alternate roll images. In either cases, o
%The difference between IPCA and IROLL is that 
IROLL directly uses the images $B_i$ obtained at roll angle $b$ as PSF model for images $A_i$ obtained at roll angle $a$ (and vice versa for the roles of $A_i$ and $B_i$) and iteratively directly removes the estimated circumstellar component from the roll images.
The difference in the IPCA algorithm is that for images $A_i$, the principal components are learned from images $B_i$ and are then projected onto the $A_i$ images to produce PSF models for subtraction (and vice versa). In practice, we used the first principal component with a cube reduced to two images corresponding to the mean image of each roll-angle observation. Thus, the only difference between IPCA and IROLL resided in the projection of a normalized mean PSF (i.e., the projected first principal component) in the former case.
    
While previous IPCA implementations considered all strictly positive residuals \citep[e.g.,][]{Pairet2021, Juillard2023}, here, we set an absolute threshold slightly above the noise level achieved in the roll-subtraction image, namely 10\,MJy/sr and 1\,MJy/sr in the F187N and F480M data, respectively. An absolute threshold like this should be used with care as it may not be appropriate for all datasets (e.g., images with a strong radial dependence on the residual noise level). We realized that this was relevant given the relatively constant (self-subtracted) noise level in the image. It was therefore particularly efficient at mitigating the propagation of ring-like artifacts \citep[observed in][]{Pairet2021, Juillard2023}.

The estimated circumstellar signal map at each iteration was smoothed using a thin 2D Gaussian kernel set to a one-pixel FWHM before its removal from the roll images that were used to produce the PSF model images at the subsequent iteration. This intends to capture the spatial correlation of neighboring pixels and leads to a better recovery of faint signals that are originally drowned in the noise level of the roll-subtraction image.
Our IPCA algorithm iteratively corrects not only for \textit{self-subtraction} because circumstellar signals are iteratively removed from the image library that is used to calculate the principal components, but also for \textit{oversubtraction}. This is because the model PSF image that is subtracted is built from the projection of principal components onto the original images minus the estimated circumstellar signal map obtained at the previous iteration.

% These algorithms lead to the best-quality images of the system  (Fig.~\ref{fig:FinalImages}b and e for the IPCA results), recovering faint circumstellar signals originally hidden in the PSF wings while iteratively correcting for geometric biases affecting extended signals \citep[see 
%     e.g.,%Fig.~1 in
%     ][]{Juillard2022}.
    
We implemented an automatic convergence criterion based on a user-defined relative tolerance (default 1e-4). When all pixel intensities in the image obtained at the subsequent iteration vary by less than this relative tolerance, the algorithm stops and considers to have converged onto a final image.
While this criterion worked for our data, likely helped by the high stability of the PSF, % of the instruments of the JWST, 
we highlight that there is no mathematical guarantee that this fix-point algorithm will systematically converge in general \citep[see also][]{Juillard2023}.
Both IROLL and IPCA converged to a final image within $\sim$1000 iterations. While IROLL recovered a significant fraction of the self-subtracted signals and yielded a similar final image as IPCA in the F187N image, this was not the case for the F480M filter image. In the latter case, it does not appear to converge onto an image free of geometric biases \citep[see e.g.,][]{Juillard2022}.

As a safety check that the algorithm properly recovered the outer disk, we show in Fig.~\ref{fig:cross-corr_IPCA_model} the Pearson cross-correlation calculated between the F187N image obtained at each iteration of IPCA and our corresponding radiative transfer model of the outer disk in a mask encompassing the south part of the outer disk (i.e.,~all the signals south of the peak intensity in Fig.~\ref{fig:disk_models}a; see Sec.~\ref{sec:NEGFD} for more details about the disk model). We highlight that this disk model is not used by IPCA, which functions in an agnostic and automated manner. Most of the geometric biases are corrected within $\sim$200 iterations. Most of the flux is also recovered within a similar number of iterations, with only marginal gains beyond $\sim$200 iterations.

\begin{figure}[h]
\centering
\includegraphics[width=\columnwidth]{Cross-corr_F187N_IPCA_vs_model.pdf}
\caption{Pearson cross-correlation coefficient calculated between the F187N image obtained by IPCA at each iteration and our radiative outer-disk model. Our disk model is not used by IPCA, but rather as a diagnostic for the efficiency of the IPCA algorithm. The geometric biases induced by roll subtraction are corrected within $\sim$200 iterations.}
\label{fig:cross-corr_IPCA_model}
\end{figure}
    
The individual and mean roll subtraction, iterative roll subtraction, and iterative PCA algorithms were all implemented in VIP and are available as of version 1.6.0. The images obtained with roll subtraction, TRAP, iterative roll subtraction, and iterative PCA are shown in Fig.~\ref{fig:AltAlgos}.

%Optimal smoothing factor found by cross-correlation. A Gaussian kernel size of **pc and 1.08px was applied to the F187N and F480M reference images, respectively.

\subsection{Reliability of the IPCA}\label{sec:TestIPCA}

    \begin{figure*}[h]
    \centering
    \includegraphics[width=\textwidth]{FigA3.pdf}
    \caption{\new{Fiducial model of the circumstellar signals of PDS 70 ({\bf d}) compared to the image recovered by IPCA ({\bf e}) using the same reduction parameters as for the PDS 70 dataset. The different components of the model are shown in panels {\bf a-c}. This model was injected at two different roll angles separated by 5.0~\degr in the NIRCam dataset of HD~135067 at a similar contrast as the circumstellar signals of PDS 70. \vc{The protoplanet locations are indicated with dashed circles in panel {\bf e}}.}
    \vc{The recovery of all signals from the model at a similar level as injected casts confidence in the results obtained with IPCA from the PDS 70 data. %\vc{Remove mask covering RA label}
    }}
    \label{fig:REF_test}
    \end{figure*}
    
\new{We tested the IPCA method on a fiducial dataset in order to illustrate its effectiveness at producing unbiased images of extended circumstellar signals, and to validate all our conclusions regarding the flux and morphology of circumstellar signals recovered in our IPCA images of PDS~70 in this way. For this test, we considered the only other JWST/NIRCam dataset obtained in the F187N filter with the NRCB1 detector and SUB64P subarray that was publicly available at the time of this study: an observation of reference star HD~135067 that was obtained on 2 February 2023 (Program 1902). This source does not have any known resolved circumstellar emission. We directly downloaded the calibrated \texttt{i2d} images from the MAST archive. While the number of integrations and the number of groups per integration were similar to the PDS~70 observation, the main difference was the adoption of a three-point dither pattern strategy instead of a five-point dither pattern for the observation of PDS~70. This results in a higher sensitivity to spatial undersampling effects, which are expected to be particularly prominent near the core of the PSF. As a consequence, the results of the test presented in this section should be considered as a conservative lower limit on the actual expected performance of IPCA on the PDS~70 dataset.}

\new{We designed a toy model for the circumstellar signals of PDS~70 that was composed of an outer disk, an inner disk, a spiral-like signal, and three point-source injections corresponding to protoplanets $b$ and $c$, and candidate $d$. For the outer disk, we considered the optimal model described in Appendix~\ref{sec:NEGFD}. For the inner disk, we considered similar parameters as presented in \citet{Keppler2018}, with the peak of the emission concentrated within $\sim$5~au. The two disk components are shown in Fig.~\ref{fig:REF_test}a. For the spiral-like signal, we considered a similar trace as observed in our F187N images of PDS~70 after subtraction of the optimal outer-disk model (Fig.~\ref{fig:REF_test}b). For the protoplanets, we considered similar positions and contrasts as inferred with NEGFC in Appendix~\ref{sec:NEGFC_optim} (Fig.~\ref{fig:REF_test}c). The sum of all components of the model is shown in Fig.~\ref{fig:REF_test}d. To create and inject the fiducial model into the reference cube, we relied on routines from the \texttt{fm} (forward-modeling) subpackage of VIP, in particular, its implementation of the Grenoble RAdiative TransfER tool \citep{Augereau1999, Milli2019} for the inner-disk model, its trace and fake-companion injection routines. Before injection in the reference cube data, the model was scaled to a similar contrast \vc{with respect to} the reference star as the circumstellar signals with respect to PDS 70. The signals were injected at two different angles separated by 5.0 deg in the two sets of three dithered images (i.e., the same roll-angle difference as in the PDS 70 observation).}

\new{Finally, we ran IPCA on this fiducial dataset using the same reduction parameters as were used to produce the F187N images of PDS~70. The algorithm converged within $\sim$ 400 iterations, resulting in the image shown in Fig.~\ref{fig:REF_test}e. To facilitate the comparison with the ground-truth injected model, we highlight the location of the injected protoplanets with dashed circles. Stellar residual artifacts can likely be assigned to small differences in the core of the PSF \vc{and spatial undersampling effects, which are more prominent when using a three-point instead of a five-point dither pattern. Apart from theses residuals,} we note a satisfactory recovery of both the morphology and flux level of the injected circumstellar signals. This makes us confident of the results obtained with IPCA on the PDS 70 dataset reported in this paper.
}

\section{Optimal disk model found with NEGFD}\label{sec:NEGFD}

    %Describe optimal radiative transfer model of the disk (NEGFD approach). 
    To be able to extract unbiased photometry for the protoplanets, the expected contribution of the disk must be removed. %This is of particular relevance for planet $c$ which lies next to the bright edge of the outer disk.
    Our goal in this work is not a full modeling of the disk, as this would involve a combined SED and disk-image fit \citep[e.g.,][]{Keppler2018,Portilla-Revelo2022, Portilla-Revelo2023}. Therefore, we decided to rely on the latest radiative transfer models of the disk produced with \texttt{MCMax3D} \citep{Min2009} and presented in \citet{Portilla-Revelo2022, Portilla-Revelo2023}, %to start with a good approximation of the outer disk, 
    as they match these combined constraints, %that are detailed in \citet{Portilla-Revelo2022, Portilla-Revelo2023}, 
    and only allow these models to vary to a small extent so that they are still compatible with ALMA and SPHERE polarized-intensity constraints. %considered in those works.
    Namely,
    % Namely, we considered the models presented in \citet{Portilla-Revelo2022, Portilla-Revelo2023}, 
    %The model leads to a good fit to observational data on the disk, such that 
    we allowed (i) the minimum grain size $a_{\rm min}$ in the grain size distribution to vary between 0.001 and 0.05 $\upmu$m; %(i.e., the shape of the total distribution is allowed to change within a small amount); 
    (ii) the settling parameter $\alpha$ to vary between 0.001\,$\upmu$m and 0.01\,$\upmu$m; (iii) spatial and flux scaling of the model image to vary within a small range around unity; and (iv) small linear (subpixel) and azimuthal (subdegree) shifts with respect to the center of the image, as these parameters are essentially constrained by imaging. Small variations in these parameters can lead to noticeable changes in the model images and therefore, to significant residuals after subtraction of the model. %, which may alter the measured flux of the two protoplanets.
    We produced a grid of ten disk models in $a_{\rm min}$ (two explored values) and $\alpha$ (five explored values), and searched for the optimal model with a Nelder-Mead algorithm by interpolating model images in log-space and including free parameters for scaling and shifts. The model images were produced at the same pixel scale as the F187N and F480M images, and they were smoothed by convolution with the observed PSF. The optimal model was then found by minimizing the absolute residuals after subtraction of the model from the observed images. 
    This procedure, which we refer to as the negative fake disk technique (NEGFD), is implemented in VIP as of version 1.6.0. 

The optimal disk model was found by minimizing the sum of absolute intensity residuals in a binary mask that encompassed the location of the outer disk while excluding the location of both planets (2-FWHM aperture exclusion masks).  
We considered two potential masks. The first mask included the whole forward-scattering (i.e., brighter) side of the disk, and the second map only included the southwest part of the disk, anticipating excess signals toward the west (near planet c) and northwest part of the disk (spiral-like feature) based on previous images of the disk \citep[][]{Wang2020, Juillard2022}. As our tests using the first mask led to a mix of strong positive and negative residuals, we only consider the results obtained by minimizing residuals in the second mask, shown in Fig.~\ref{fig:disk_models}c, in the rest of this work. 
    In practice, we identified the optimal radiative transfer disk model using the F480M data because of the higher signal-to-noise ratio of the disk in these data. The optimal $a_{\rm min}$ and $\alpha$ values were found to be 0.001\,$\upmu$m and 0.01, respectively, meaning that the optimal \vc{model} is in between the radiative transfer models considered in \citet{Portilla-Revelo2022} and \citet{Portilla-Revelo2023}. 
    These parameters were then used to make a disk model prediction at 1.87\,$\upmu$m. %, which is subsequently subtracted from the F187N data after convolution with the synthetic PSF of the instrument.
    The best F187N and F480M models are shown in Fig.~\ref{fig:disk_models}a and b, respectively. %, along with the mask used for minimization. %in the F480M images. 
    
    %**Discuss reason it could hit the edge of the grid. It does not matter for our purpose. Flaring exponent is fixed (1.14) but may be too small \citep[][]{} , which is likely compensated with a higher turbulent parameter. As these may be degenerate, but our goal is only to reproduce the observed disk signal of a compatible with the SED and other observational constraints.  We argue exploring these two parameters are sufficient for our goal of highlighting both protoplanet and strong non-symmetric disk signals, with the caveat that our final values of $a_{\rm min}$ and $\alpha$ should be taken with a pinch of salt.

    \begin{figure*}[!t]
    \centering
    \includegraphics[width=\textwidth]{FigB2.pdf}
    \caption{Best-fit radiative transfer disk models for the (a) F187N and (b) F480M images, and (c) optimization mask used to determine the optimal F480M disk model. %\vc{Remove mask covering RA label}
    }
    \label{fig:disk_models}
    \end{figure*}

Figs.~\ref{fig:NEGFD_and_NEGFC}b and e show the roll subtraction images obtained after subtracting the optimal disk model found by NEGFD from the original images (i.e., before PSF subtraction) for the F187N and F480M datasets, respectively. These are compared to the original roll subtraction images (Figs.~\ref{fig:NEGFD_and_NEGFC}a and d) using the same intensity scale. The disk subtraction performs well, in particular, for the southwest part of the disk, where the residuals reach a similar level as the residual noise level in the image. Subtraction of the optimal disk model clearly highlights the presence in the F480M image of the arm-like feature characterized in \citet{Juillard2022} in VLT/SPHERE images. This feature is indicated with a thick arrow.
We also checked how a similar NEGFD procedure performed on the IPCA images obtained with both filters by fixing the $a_{\rm min}$ and $\alpha$ to the optimal values found by NEGFC+roll subtraction. The results are shown in Fig.~\ref{fig:FinalImages}b and e, and they are discussed in Sec.~\ref{sec:spiral}.


\section{NEGFC retrieval of planet parameters} \label{sec:NEGFC_optim}

The NEGFC technique consists of finding the optimal parameters (radial separation, position angle, and contrast with respect to the star) of a directly imaged companion through the injection of fake companions with a negative flux in the input image cube (i.e., before post-processing), such that residuals are minimized in the post-processed image around the location of the planet. This forward-modeling approach is necessary to avoid the parameter estimates for the companion to be affected by self- and oversubtraction effects inherent to the post-processing algorithm used.

We used the Nelder-Mead minimization algorithm implemented in VIP and 
described in \citet{Wertz2017} to estimate the parameters for the protoplanets.
This NEGFC implementation %in VIP 
offers different options in terms of figure of merit to be used in the processed image after injection of negative fake companions to identify the optimal radial separation $r_p$, position angle PA$_p$, and contrast $f_p$ of a given planet, namely minimizing (i) the sum of absolute intensity residuals in an aperture encompassing the companion, (ii) the standard deviation of intensity residuals in an aperture encompassing the companion, or (iii) the sum of absolute determinant values of the Hessian matrix calculated for each of the $n_d \times n_d$ pixels surrounding the planet location. The latter option is equivalent to minimizing the local absolute curvature, and it is more appropriate for the extraction of point-source astrometry and photometry in the presence of underlying extended signals \citep[e.g.,][]{Quanz2015}. We implemented it for this work and made it available in VIP as of version 1.6.0. % outdated: We selected the standard deviation figure of merit in either a 1-FWHM or 2-FWHM aperture centered around a first guess estimate of each planet. We justify this choice to minimize the risk of contamination of inner disk signal for planet $b$ and residual outer disk signal for planet c, in their respective flux estimate. Another advantage of the standard deviation figure of merit, we expect our planet flux estimates to be less dependent on the accuracy of the outer disk model that is subtracted. On the other hand, our tests with a figure of merit minimizing absolute residual intensities in the final image show systematically larger estimates in flux by 30\% and lower estimates on the estimated radial separation of planet $b$ ($\lesssim$ 115mas), suggesting non-negligible contamination by poorly resolved inner disk signals. **We discuss the determination of optimal figure of merit and  aperture size in Appendix**. Selected between 1FWHM and 2FWHM based on the minimization of uncertainties after injection of 360 fake positive companions, and minimization of the determinant of the Hessian matrix for planet $b$ in the F480M data.
%Compared to most works involving the NEGFC technique, we implemented a new figure of merit which consists in 
    %The main difference with previous works involving the NEGFC technique is the use of a figure of merit consisting in 
    %minimizing the absolute sum of the determinant of the Hessian matrices associated with the pixels surrounding the location of each protoplanet
    %This is equivalent to minimizing the local absolute curvature, and is more appropriate for the extraction of point source astrometry and photometry in the presence of underlying extended signals \citep[e.g.,][]{Quanz2015}.
We detail two different approaches below that involve NEGFC that we tested and compared to retrieve optimal astrometry and photometry for the protoplanets.
    
\subsection{Classical forward-modeling NEGFC}

    \begin{figure*}[!t]
    \centering
    \includegraphics[width=\textwidth]{FigB1.pdf}
    \caption{Images obtained at 1.87$\upmu$m (F187N; top row) and 4.80$\upmu$m (F480M; bottom row) after subtraction of the stellar PSF using roll subtraction (first column), and after additional subtraction of the best outer-disk model using the negative fake disk technique (second column; details in Appendix~\ref{sec:NEGFD}). The third column shows the %signal-to-noise ratio map 
    residuals after both disk and planet subtraction, using the parameters found with NEGFC (Appendix~\ref{sec:NEGFC_optim}). The circles indicate the predicted location of the two protoplanets based on the orbital fits presented in \citet{Wang2021}.
    }
    \label{fig:NEGFD_and_NEGFC}
    \end{figure*}

We first considered the classical forward-modeling NEGFC approach as described in \citet{Lagrange2010} or \citet{Wertz2017}, combined with roll subtraction, hereafter referred to as NEGFC+roll. We applied NEGFC+roll to the image cube where the optimal disk model found by NEGFD was removed (Appendix~\ref{sec:NEGFD}).
Regarding the choice of figure of merit, we note that different figures of merit lead to more or less reliably retrieved planet parameters depending on the S/N of the companion, the nature of the local noise, and contamination by residual extended signals (e.g., inner disk or spiral accretion stream signal). We therefore determined the optimal figure of merit on a case-by-case basis for each planet in each of our two datasets as follows. %, hence its radial separation.
%We proceeded as follow to derive the optimal figure of merit. 
We used NEGFC to derive first estimates on the parameters of the companion with each of the three figures of merit, considering two subcases for the sum and standard deviation, corresponding to their calculation in either 1-FWHM or 2-FWHM size apertures, and three subcases for the Hessian-matrix determinant figure of merit, corresponding to its calculation with $n_d$ set to 1, 2, or 3. Then we injected 360 fake (positive) companions with the radial separation and contrast inferred by NEGFC, 1$\degr$ apart from each other in separate copies of the original datasets, and individually retrieved their parameters for each NEGFC subcase.
%For the number of pixels for which the Hessian matrices are calculated, hence their determinants, we set it automatically based on the angular resolution of the observations and radial separation of the companion, as follow: $n_{\rm det}$ = $2 \times \max(\min({\rm FWHM}, r_{fg}), 2)$. The sum figure of merit systematically overestimated the contrast. 
Finally, we considered the subcase that led to the smallest deviations between the retrieved and injected fake companion parameters. These deviations were also used as residual noise uncertainty on the retrieved parameters.
%We estimated conservative uncertainties on the radial separation, PA and contrast of the planets with respect to the star through the injection of 360 fake (positive) companions in the image cube before roll subtraction, and using the parameters found with the simplex. We then used the distribution of deviations between retrieved and injected parameters to estimate the uncertainties. This approach captures the uncertainty associated to the level of stellar subtraction residuals in the image at the radial separation of each planet.

We did not retrieve the parameters for planet $c$ or candidate $d$ in the F187N data because they are not detected at a significant level in %has a very low S/N in 
the F187N roll-subtracted image. 
For all other cases, namely planet $b$ in the F187N and F480M datasets and planet $c$ in the F480M dataset, we note that the Hessian figure of merit outperformed the other two figures of merit in terms of accuracy of the inferred astrometry. It also retrieved better the contrast of the injected companions than the sum figure of merit, which tended to overestimate their flux, likely due to residual extended signals.
%For PDS~70~$b$, the measured flux and position may be contaminated by the inner disk due to the lack of angular resolution to disentangle both contributions (see e.g., Fig.~\ref{fig:FinalImages}). This potential bias is mitigated by using the Hessian figure of merit. 
For planet $b$ in the F480M data, the Hessian figure of merit is the only metric that did not significantly underestimate the radial separation of the planet compared to its expected location from the orbital fits presented in \citet{Wang2021}. The poorer performance of other metrics may be assigned to diffuse inner-disk emission (see, e.g.,~the IPCA images in Fig.~\ref{fig:FinalImages}), pushing the inferred centroid toward the center. Minimizing the determinant of the Hessian matrix is also particularly appropriate for planet c because it involves a lower sensitivity to the exactitude of the disk model inferred with NEGFD, which is subtracted from the cube before inferring the parameters of the planet. $n_d$ set to 2 or 3 led to consistent results for the Hessian figure of merit, while $n_d = 1$ was more prone to yielding outlier values.
% For PDS~70~$b$, the measured flux may be contaminated by the inner disk due to the lack of angular resolution to disentangle both contributions. This potential bias is mitigated by minimizing the sum of the determinant of Hessian matrices figure of merit. This is suggested by the shorter radial separation inferred by the NEGFC procedure in the F480M images compared to the F187N dataset, and the even shorter radial separation using the sum of absolute residual intensities figure of merit for the NEGFC minimization instead of the standard deviation. 
%Potential test? MCMC exploration: based on preliminary tests, this should show an anti-correlation between inferred flux and radial separation.
%We did not retrieve the parameters for planet $c$ in the F187N data, as it is not detected at a significant level in %has a very low S/N in 
%the F187N roll subtracted image. %In this case, we fixed the position of the companion in the NEGFC procedure to either the one inferred in the F480M dataset (as its location is unclear in the image) or at the predicted position of the planet based on the orbital fits presented in \citet{Wang2021}. We then only searched for the optimal contrast at those positions for the different figures of merit. Either assumptions for the position of the planet led to consistent results for the extracted flux. %For this specific case, we only searched for the optimal contrast at those positions for the different figures of merit. %We adapted the fake companion injection test accordingly to only retrieve the contrast.
%In this case, the standard deviation figure of merit calculated in a 1-FWHM aperture led to the smallest deviations in the (positive) fake companion injection test. %At the contrast of the planet, 
%Again, this was determined by the injection of 360 (positive) fake companions, the retrieval of their parameters using NEGFD, and inspection of the distribution of the deviations with the ground truth parameters \citet[as done in][]{Wertz2017}. For all tests, the deviations followed roughly a normal distribution, hence we considered the standard deviation the distribution as uncertainty for each parameter.

The final uncertainties combine the residual noise uncertainties estimated with the positive fake companion injection test in quadrature with systematic uncertainties for $r_p$ and PA$_p$ and with photon-noise \vc{uncertainties} for $f_p$. We considered systematic astrometric uncertainties of 6 mas and 2 mas, for the F187N and F480M data, respectively, based on the JWST user manual. The uncertainties associated with the stellar flux are negligible in comparison to the other contributions. The results are reported in the `Roll subtraction' column of Table~\ref{tab:comparison_NEGFC}.

As a cross check, we examined the roll-subtracted images obtained after subtraction of both the disk model (NEGFD) and of the planet signals using the values retrieved by NEGFC. This is shown in the last column of Fig.~\ref{fig:NEGFD_and_NEGFC}. The residuals at the expected (circled) location of the planets are consistent with the local noise level.

\begin{table*}
\begin{center}
\caption{Comparison of the properties of the protoplanets inferred from the NIRCam F187N and F480M observations using NEGFC combined with either roll subtraction or IPCA, and the predictions made in either \citet{Wang2021} for $b$ and $c$ and \citetalias{Mesa2019a} for candidate $d$.} 
\label{tab:comparison_NEGFC}
\begin{tabular}{lcccc}
\hline
\hline
Parameter & Roll subtraction & IPCA & Final & Prediction$^{\rm (a)}$\\
\hline
\multicolumn{5}{c}{F187N} \\
\hline
\multicolumn{5}{c}{PDS~70~$b$}\\
\hline
Separation$^{\rm (b)}$ & $164.5 \pm 12.0$ & $151.9 \pm 12.1$ & $158.2 \pm 12.1$ & $155.5 \pm 1.4$ %& $145.6 \pm 20.4$ & * & $155.5 \pm 1.4$ 
\\
PA$^{\rm (c)}$ & $129.0 \pm 4.1$ & $129.4 \pm 2.6$ & $129.2 \pm 4.1$ & $132.6 \pm 0.3$ %& $130.4 \pm 6.7$ & * & $132.6 \pm 0.3$ 
\\
Contrast$^{\rm (d)}$ & $(3.28 \pm 0.78) \times 10^{-4}$ & $(2.50 \pm 0.56) \times 10^{-4}$ & $(2.89 \pm 0.78) \times 10^{-4}$ & $(2.4 \pm 0.1) \times 10^{-4}$ %& $(3.93 \pm 1.11) \times 10^{-3}$ & $(3.3 \pm *) \times 10^{-3}$ & * 
\\
Flux$^{\rm (e)}$ & $(8.69 \pm 2.07) \times 10^{-17}$ & $(6.62  \pm 1.48) \times 10^{-17}$ & $(7.65  \pm 2.07) \times 10^{-17}$ & $(6.3  \pm 0.3) \times 10^{-17}$ %& *$(7.6 \pm 2.1) \times 10^{-17}$ & * & *  % Poisson unc: ~1.1%: negligible compared to 20%
\\
\hline
\multicolumn{5}{c}{PDS~70~$c$}\\
\hline
Separation$^{\rm (b)}$ & -- & $203.3 \pm 23.3$ & $203.3 \pm 23.3$ & $210.1 \pm 1.0$ %& $216.7 \pm 20.6$ & $219.2 \pm 11.0$ & $210.1 \pm 1.0$
\\
PA$^{\rm (c)}$ & -- & $269.7 \pm 6.8$ & $269.7 \pm 6.8$ & $270.1 \pm 0.3$%& $272.8 \pm 3.6$ & $278.7\pm3.5$ & $270.1 \pm 0.3$ 
\\  %7.0 instead of 3.5
Contrast$^{\rm (d)}$ & -- %$(8.8 \pm 6.4) \times 10^{-5}$ 
& $(1.37 \pm 0.42) \times 10^{-4}$ & 
$(1.37 \pm 0.42) \times 10^{-4}$
%$(1.12 \pm 0.64) \times 10^{-4}$
& $(4.5 \pm 0.4) \times 10^{-5}$ %& $(1.55 \pm 0.30)  \times 10^{-3}$ & $(1.4 \pm 0.2) \times 10^{-3}$ & *
\\
Flux$^{\rm (e)}$ & -- %$(2.33 \pm 1.69) \times 10^{-17}$ 
& $(3.63 \pm 1.11) \times 10^{-17}$ & 
$(3.63 \pm 1.11) \times 10^{-17}$
%$(2.96 \pm 1.69) \times 10^{-17}$ 
& $(1.2 \pm 0.1) \times 10^{-17}$ %& * & * & *
\\
\hline
\multicolumn{5}{c}{PDS~70~d?}\\
\hline
Separation$^{\rm (b)}$ &  -- & $103.4 \pm 23.2$ & $103.4 \pm 23.2$ & $90.4 \pm 5.9$ %& -- & -- & --
\\
PA$^{\rm (c)}$ & -- & $293.0 \pm 12.7$ & $293.0 \pm 12.7$ & $284.6 \pm 1.3$%& -- & -- & -- 
\\  %7.0 instead of 3.5
Contrast$^{\rm (d)}$ & -- & $(1.88 \pm 0.69) \times 10^{-4}$ & $(1.88 \pm 0.69) \times 10^{-4}$ & $(1.2 \pm 0.2) \times 10^{-4}$ %&  -- & -- & --
\\
Flux$^{\rm (e)}$ & -- & $ (5.25 \pm 1.93) \times 10^{-17}$ &  $(5.25 \pm 1.93) \times 10^{-17}$ & $(3.4 \pm 0.6) \times 10^{-17}$ %& -- & -- & --
\\
\hline
\multicolumn{5}{c}{F480M}\\
\hline
\multicolumn{5}{c}{PDS~70~$b$}\\
\hline
Separation$^{\rm (b)}$ & $145.6 \pm 20.4$ & $141.6 \pm 10.9$ & $143.6 \pm 20.4$ & $155.5 \pm 1.4$ \\
PA$^{\rm (c)}$ & $130.4 \pm 6.7$ & $127.1 \pm 6.7$ & $128.7 \pm 6.7$ & $132.6 \pm 0.3$ \\
Contrast$^{\rm (d)}$ & $(3.93 \pm 1.50) %new conservative unc.
%1.11) % old unc
\times 10^{-3}$ & $(2.96 \pm 1.02) %new conservative unc.
%0.52) old unc
\times 10^{-3}$ & $(3.44 \pm 1.50) %new conservative unc.
%1.11) % old unc
\times 10^{-3}$ & $(2.8 \pm 0.6) \times 10^{-3}$ \\
Flux$^{\rm (e)}$ & $(8.13 \pm 3.11) %new conservative unc.
%2.30) % old unc
 \times 10^{-17}$ & $(6.12 \pm 1.08) \times 10^{-17}$ & $(7.11 \pm 3.11) %new conservative unc.
%2.30) % old unc
\times 10^{-17}$ & $(5.8 \pm 1.1) \times 10^{-17}$\\
\hline
\multicolumn{5}{c}{PDS~70~$c$}\\
\hline
Separation$^{\rm (b)}$ & $216.7 \pm 20.6$ & $219.2 \pm 12.7$ & $217.9 \pm 20.6$ & $210.1 \pm 1.0$\\
PA$^{\rm (b)}$ & $272.8 \pm 3.6$ & $276.8\pm3.7$  & $274.8\pm3.7$ & $270.1 \pm 0.3$ \\  %7.0 instead of 3.5
Contrast$^{\rm (c)}$ & $(1.55 \pm 0.30)  \times 10^{-3}$ & $(1.39 \pm 0.15) \times 10^{-3}$ & $(1.47 \pm 0.30)  \times 10^{-3}$ & $(1.0 \pm 0.3) \times 10^{-3}$\\
Flux$^{\rm (e)}$ & $(3.21 \pm 0.62) \times 10^{-17}$ & $(2.87 \pm 0.62) \times 10^{-17}$ & $(3.04 \pm 0.62) \times 10^{-17}$ & $(2.1 \pm 0.6) \times 10^{-17}$\\
\hline
\end{tabular}
\end{center}
Notes: $^{\rm (a)}$Astrometric and contrast or flux predictions based on orbital fits and best-fit atmospheric models presented in \citet{Wang2021} and \citetalias{Mesa2019a}. The contrast value reported for $d$ in \citetalias{Mesa2019a} is corrected for the expected change in total intensity-scattering efficiency based on the difference in PA with respect to the 2019 epoch. $^{\rm (b)}$Radial separation in mas. $^{\rm (c)}$Position angle measured east of north in $\degr$. $^{\rm (d)}$Contrast ratio with respect to the star and unresolved inner disk.
$^{\rm (e)}$Spectral flux density in Wm$^{-2} \upmu$m$^{-1}$. %For the IPCA estimates, the figure of merit used is indicated: \textit{sum} for the absolute sum of intensity residuals, and \textit{H} for the sum of absolute values of the Hessian matrix determinants. %$^{\rm (f)}$Median and standard deviation of values found with 5 different figures of merit: Hessian determinant minimization considering either the closest pixel, the 2x2 surrounding pixels, or the 3x3 surrounding pixels, and standard deviation minimization in either 0.5 or 1 FWHM-size apertures centered on the source. 
\end{table*}


\subsection{Direct NEGFC on the IPCA images} 

We also considered NEGFC performed directly on the IPCA images, hereafter NEGFC+IPCA, assuming that self- and oversubtraction effects are (close to) fully corrected for in these images because the IPCA images also recovered significant extended signals. It is unclear what the optimal figure of merit should be for protoplanet signals from close to or overlapping with complex extended signals, such as an inclined inner disk or a spiral accretion stream. The main source of uncertainty for the parameters of the protoplanets for NEGFC+IPCA is therefore rather associated with the correct assumption to be made in terms of the contribution from underlying extended signals at the location of the protoplanets. We therefore considered a range of figures of merit and associated parameters,
%We considered the median of results obtained with 8 slightly different combinations of figures of merit and associated parameters, which generally corresponded
%The different cases 
corresponding to using either the sum, standard deviation, or Hessian-matrix determinant figures of merit for a source position corresponding to either the predicted location based on \citet{Wang2021} and \citetalias{Mesa2019a} orbital fits or the visual location of a local maximum in intensity in the image. We considered two subcases corresponding to using either 1-FWHM or 2-FWHM apertures for the sum and standard deviation figures of merit, and three subcases for using either $n_d$ = 1, 2, or 3 %encompassing the considered position 
for the Hessian-based figure of merit.
We thus retrieved planet parameters for 14 cases in total for each protoplanet in each dataset.

These 14 cases we considered in our estimate of the final planet parameters were vetted visually upon inspection of the IPCA image after subtracting a negative planet model with parameters determined by the different figures of merit. 
We only considered visually pleasing results after subtracting the estimated protoplanet parameters from the image considering our prior knowledge of the disk (e.g., inner and outer-disk geometry), such that valid estimates typically were those from which the local intensity peak was removed while not creating a significant hole within the surrounding extended signal distribution.
The selection of different cases depended on the considered protoplanet and filter because each image and protoplanet location are affected by more or fewer surrounding circumstellar signals.
% We checked that uncertainties found with such method were consistent with what from visual inspection of NEGFC-IPCA images. This is shown in Fig.**
We then considered the median of the results obtained for the cases validated visually and adopted the standard deviation of these results as the uncertainty associated with the contamination by surrounding extended signals.
For the classical NEGFC approach, we added systematic astrometric uncertainties and photon noise uncertainties in quadrature to the uncertainties associated with contamination by surrounding extended signals.
Our results are reported in the `IPCA' column of Table~\ref{tab:comparison_NEGFC}.

% Compared to NEGFC+roll, NEGFC+IPCA could estimate more reliable parameters for planet $c$ and candidate $d$ in the F187N data, given their visual presence in the IPCA images, while their signal was self-subtracted down to a similar level as residual noise in the roll-subtracted image. %, and 
% This impeded any parameter estimate with NEGFC+roll for candidate $d$ altogether. 
%Nonetheless, g
Because of the overlap between the signals of planet $c$, candidate $d$, the spiral accretion stream, and the inner disk signals, we adopted an iterative joint-fitting strategy to estimate the parameters for planet $c$ and candidate d. We started by estimating the parameters from $c$ because it is more easily dissociated from surrounding extended signals, estimated the parameters for d in the image without the signal of $c$ (similar to Fig.~\ref{fig:FinalImages}c), then reestimated the parameters for $c$, this time, in an image from which the estimated contribution of d was removed, and so on. This procedure converged to stable estimated parameters for both sources within five iterations. We conservatively adopted for both sources the astrometric uncertainties corresponding to the largest uncertainty found among the two sources.
%We did not explore a joint fitting of both sources. Although this approach could also be followed, it would have faced a similar difficulty regarding the dependency of results to the choice of a meaningful figure of merit for a larger area encompassing structured extended signals (i.e., a spiral accretion stream and inner disk signals).

As a safety check, we inspected the IPCA images after subtracting the optimal parameters found for the protoplanets with the direct NEGFC procedure using the image cube from which the optimal outer-disk model was subtracted (NEGFD) to ensure that the estimates did not over- or underestimated the flux. 
%This is shown in Fig.~\ref{fig:FinalImages}c and f for the subtraction of planets $b$ and $c$ in the F187N and F480M images, and in Fig.~\vc{TBD}%\ref{fig:NEGFD_and_NEGFC}
%for the additional subtraction of candidate $d$ in the F187N image.
%We see that the only outstanding signals left correspond to the inner disk and what we interpret as the spiral accretion stream feeding the circumplanetary disk of protoplanet $c$. %compares the F187N and F480M images reduced with roll subtraction (panels a and d) to the residuals after successive subtraction of the optimal disk model found with the NEGFD procedure (panels b and e) and optimal planet parameters found with the NEGFC procedure (panels c and f).

    \begin{figure*}[!t]
    \centering
    \includegraphics[width=\textwidth]{FigC2.pdf}
    \caption{Images obtained at 1.87$\upmu$m (F187N; first column) and 4.80$\upmu$m (F480M; middle and right columns) before (top row) and after (bottom row) subtraction of point-source models at the location of protoplanets $c$ (first and last column) and $b$ (middle column) using fluxes \textit{different} than the optimal ones found with NEGFC. We test whether the predicted flux for planet $c$ with the DRIFT-PHOENIX model presented in \citet{Wang2021} could account for the observed flux in the F187N (panels a vs. d) and F480M (panels c vs. f) images at the location of $c$. We also show the F480M image obtained after subtracting the lower uncertainty limit from the optimal flux found by NEGFC for planet $b$ (panels b vs. e), which illustrates our uncertainty associated with the underlying inner-disk signal contribution.
    }
    \label{fig:NEGFC_safety_checks}
    \end{figure*}

\subsection{Final NEGFC results} 

Table~\ref{tab:comparison_NEGFC} shows that the astrometry and photometry extractions using NEGFC+roll and NEGFC+IPCA are consistent in all cases when retrieval of planet parameters could be made in both the roll-subtracted and IPCA images (i.e., planet $b$ with both filters, and planet $c$ in the F480M data). This suggests that IPCA recovered most of the self- and oversubtracted flux affecting roll-subtracted images. Nonetheless, neither approach is devoid of weakness for the parameter estimation. It is unclear whether IPCA did recover \textit{all} the self- or oversubtracted flux for cirumstellar signals, including the protoplanets, while on the other hand, it is unclear how much residual extended signals filtered by roll subtraction may lead to a misestimation of the NEGFC+roll planet fluxes. We therefore conservatively consider (presented in Table~\ref{tab:results}) the mean of the results obtained with NEGFC+roll and NEGFC+IPCA as our final results when both estimates are available, and we adopt the larger uncertainties of the two approaches. %We discuss individual cases below.

%We discuss below the planet parameters retrieved by each approach.
%For planet $b$ the estimates are consistent: $b$ in F187N (with std figure of merit, potential diffuse dust emission surrounding the planet...) and F480M (with std figure of merit, in aperture radius set to radial separation).

%Compared to NEGFC+roll, NEGFC+IPCA could estimate parameters for planet $c$ and candidate $d$ in the F187N data, given their visual presence in the IPCA images. %, while their signal was self-subtracted down to a similar level as residual noise in the roll-subtracted image. %, and 
%This self-subtraction impeded any parameter estimation with NEGFC+roll for candidate $d$ altogether. As discussed in the previous subsection, the uncertainties associated to the astrometry of $c$ and $d$ are similar due to the overlap in their signals, and the adopted iterative joint-fitting procedure.

In general, the visual vetting of NEGFC+IPCA results rejected the results obtained with the sum figure of merit as it systematically led to a hole within the surrounding patch of pixels, which we interpret as likely overestimating the flux of the protoplanet alone. Only in the case of planet $b$ in the F187N data did the results appear visually satisfactory because the protoplanet is located just outward of the inner disk (although the signals from the two components appear to be connected at the resolution of our observations). In this case, it was fair to consider the possibilities that either most of the flux at the location of $b$ within our flux uncertainties is due to the protoplanet itself, or that there is nonzero contributing signal from the inner disk. These possibilities are well captured with the sum and standard deviation figures of merit, respectively. We therefore considered the median and standard deviation of the parameters retrieved for all corresponding subcases %involving them 
as our reported NEGFC+IPCA results. We also note that in this case (planet $b$ in F187N) alone, the Hessian-based figure of merit did not work properly for NEGFC+IPCA because the flux levels for $b$ and the adjacent inner-disk signals were similar.
In contrast, this figure of merit worked the best in the NEGFC+roll approach compared to the sum or standard deviation figures of merit, based on our positive fake companion injection tests, which we interpret as due to the stronger self-subtraction of (the closer-in) inner-disk signals, which causes the planet signal to stand out from it. %The stronger self-subtraction of inner disk signals in the roll-subtracted images likely worked in favour of the Hessian-based figure of merit in the NEGFC+roll approach. %We therefore minimized instead the classical figure of merit consisting in the sum of absolute intensity residuals in a 1-FWHM aperture. We note that these inner disk signals being much more self-subtracted in the roll-subtraction images, the Hessian-based figure of merit worked well for that algorithm.
Similar remarks can be made for planet $b$ in the F480M images, where the IPCA images recover inner-disk signals better and therefore tend to provide slightly closer radial separation estimates for the protoplanet due to the bias from the inner disk. %although the planet signal surpasses surrounding inner disk signals, which makes Hessian-based NEGFC+IPCA retrieval more reliable than in the F187N case, but only with $n_d = 1$ given the close separation to the star.
%In that case, we adopted the results obtained with the Hessian-based figure of merit in the NEGFC+roll approach for our final reported results. 

For planet $c$ in the F187N data, the estimated parameters are affected by large uncertainties %with either approaches, % in both the roll subtraction and IPCA images 
%due to different reasons. 
%In the roll-subtracted image, most of the planet signal is self-subtracted down to the residual noise level, while on the contrary the IPCA image recovered 
because many additional signals are recovered by IPCA around the planet, including the spiral accretion stream and candidate $d$, which makes the estimate of the contribution from the planet alone difficult. %The uncertainties with the latter approach being smaller than with , where the planet is barely seen at a similar level as the noise, we considered the NEGFC+IPCA estimates for our final reported parameters in Table~\ref{tab:results}.
%Compared to NEGFC+roll, NEGFC+IPCA could better estimate the parameters of planet $c$ and candidate $d$ in the F187N data, given their visual presence in the IPCA images, while their signal was drowned down to the noise level in the roll subtracted image. We conservatively consider the largest astrometric uncertainties for both , and converted it into uncertainties in $r_p$
%For planet $b$ in the F480M images, while the NEGFC+roll and NEGFC+IPCA approaches led to consistent estimates, we favour the slightly lower flux estimate obtained with the NEGFC+IPCA approach. In this case, the planet signal peaks above the neighbouring inner disk signals, such that either the Hessian-based or standard-deviation based figures of merit lead to meaningful results.
For planet $c$ in the F480M data, the NEGFC+roll and NEGFC+IPCA results are very similar. Both approaches lead to visually satisfactory results in the images obtained after subtracting the respective optimal planet parameters. %, for the considered figures of merit. %(Figs.~\ref{fig:NEGFD_and_NEGFC}f and \vc{TBD}).
We nevertheless note a potential shift to a larger estimated PA in the F480M data than expected from orbital fits, which may either reflect contamination by the spiral accretion stream in this lower angular resolution image or a misestimation of the orbit from earlier astrometric measurements.

%For planet $c$ in the F187N IPCA image, we used the Hessian-based figure of merit, as it was appropriate to the presence of significant surrounding signals, from the inner disk and the overlapping spiral accretion stream. This lead to consistent results as obtained with the standard deviation figure of merit used in the case of roll subtraction (which was the most relevant, given the tentative presence of the companion at a level commensurate with the noise in that image).

% Results: median of 8 subcases. Uncertainties associated to surrounding extended signals are considere to the standard deviation of the results obtained in these 8 subcases. Final uncertainties combine those uncertainties in quadrature with systematic uncertainties for $r_p$ and $PA_p$, and with photon noise uncertainties for $f_p$. The uncertainties associated to the stellar flux are negligible in comparison to other contribution.

% The final reported astrometry and photometry of the protoplanets correspond to the results obtained with the NEGFC procedure applied to the roll subtraction images, instead of the IPCA images, as the former naturally removes the azimuthally extended flux contributions, hence mitigates the risk of including resolved disk signals. We nonetheless also tested astrometry and photometry extraction through NEGFC directly applied to the final IPCA images (which we hereafter refer to as \textit{direct} NEGFC procedure, as opposed to the classical \textit{forward-modeling} NEGFC). The forward modeling approach is indeed poorly suited to the iterative process involved in IPCA. Table~\ref{tab:comparison_NEGFC} compares the astrometry and photometry inferred with the forward-modeling and the direct NEGFC approaches, combined with roll subtraction and IPCA, respectively.  
% Both approaches lead to overall concurring estimates. This suggests that IPCA recovers most of the self-subtracted flux affecting either roll subtraction or (non-iterative) PCA results.
% Below we detail the specific choices made in terms of figure of merit, which is relevant to the case of each protoplanet in each dataset.

% Our final uncertainties also include systematic astrometric uncertainties and photon noise, added in quadrature with the uncertainties obtained based on our injection test of 360 fake companions (NEGFC+roll case), or the scatter of values obtained for different relevant figures of merit (NEGFC+IPCA case). The systematic astrometric uncertainties are of the order of $\sim$\,6 mas and $\sim$2\,mas for the F187N and F480M filters, respectively, based on the JWST user documentation.

% A marginally lower separation than expected is found for planet $c$ in the F187N. We note that the estimate is complicated by the presence of overlapping extended signals. If such lower separation is confirmed, it may be due to part of the signal tracing scattered light, hence coming potentially higher than the planet. Since a circumplanetary disk was detected around c, it is conceivable that part of the signals traces scattered light by dust higher up, with such effect accentuated from the planet approaching the semi-minor axis on the near side of the disk.


% \subsection{Full forward modeling of the disk and planets}

% Alternatively to consecutive application of NEGFD and NEGFC, we also considered a full forward modeling approach, with different free parameters corresponding to the disk and protoplanets, and convolution with either a model PSF or the observed PSF of the star after subtraction of the optimal disk and planet models found with the NEGFD and NEGFC procedures...***To be filled by Matthias...

\subsection{Additional tests}

To determine excess signals compared to atmospheric model predictions for the protoplanets (Sec.~\ref{sec:photometry}), we subtracted a scaled version of the observed PSF with the appropriate contrast ratio to match the flux predicted by the best-fit Drift-Phoenix model for the SED of planet $c$ presented in \citet{Wang2021} at the predicted location for the planet based on the orbital fits presented in \citet{Wang2021}. The position inferred by NEGFC in the previous sections instead of the predicted location does not change the conclusion. 
The results of this safety check are shown in the left and right columns of Fig.~\ref{fig:NEGFC_safety_checks} for PDS~70~c in the F187N and F480M filter images, respectively. The residual signals observed at the location of the protoplanet indeed suggest excess emission compared to predictions from the best-fit atmospheric models alone.

We performed a similar test for planet $b$ in the F480M data, shown in the middle column of Fig.~\ref{fig:NEGFC_safety_checks}. In this case, we subtracted the flux corresponding to the lower uncertainty of the estimate found with NEGFC (reported in Table~\ref{tab:results}), which is roughly midway between the best-fit BT-Settl model with blackbody excess and both the Drift-Phoenix and extinct EXOREM models without blackbody excess. This test suggests that our reported uncertainty is sound, and that there is thus tentative excess compared to predictions from atmospheric models without an additional blackbody contribution representative of circumplanetary disk emission. %, as any further excess would appear more likely assigned to the protoplanet, than to underlying inner disk signal.

For comparison, the F187N and F480M images obtained after subtracting protoplanets $b$ and $c$ with the optimal parameters found by NEGFC are shown in Fig.~\ref{fig:FinalImages}c and f.

% Second series of tests:
% Inject a fake planet with the F187N flux estimated in IPCA on the opposite side, and perform roll subtraction. Does the companion get self-subtracted to a similar level as the real c?


\section{Predicted and observed spirals}

    \begin{figure*}
    \centering
    \includegraphics[width=\textwidth]{FigC_v2.pdf}
    \caption{Comparison between the hydrodynamical simulation of the system in \citet{Toci2020}, who predicted a spiral accretion stream feeding planet $c$ (left), and our observations in the F187N (middle) and F480M (right) filters. The trace of the spiral accretion stream is identified as local radial maxima in the simulation and is shown as white dots in all images. The trace of the outer arm characterized in \citet{Juillard2022} is shown with orange dots. The part of the simulated image that is affected by subtraction artifacts is masked to show the relevant part of the image. %\vc{TBD: add fork symbol in c) and reduce circles size in b)}
    }
    \label{fig:SpiralTrace}
    \end{figure*}


Figure~\ref{fig:SpiralTrace} compares our disk-subtracted F187N and F480M images (panels b and c) to a simulated IR image predicting the spiral accretion stream associated with protoplanet $c$ (panel a). The latter is based on dedicated 3D hydrodynamical simulations made with the smoothed-particle hydrodynamics code PHANTOM \citep{Price2018}, presented in \citet{Toci2020}. The simulated image corresponds to a %MCFOST 
radiative transfer prediction at 2.11\,$\upmu$m of the PHANTOM simulation made with the Monte Carlo radiative transfer code MCFOST \citep{Pinte2006}, 
where a proxy for the outer-disk signal was subtracted. This proxy consists of another radiative transfer prediction for a different snapshot of the same hydrodynamical simulation, corresponding to planet $c$ being located at 180\,$\degr$ from its observed position angle. We therefore masked the bottom part of the image because it is affected by strong residuals associated with the spiral accretion stream that was subtracted from that part of the image. We measured the trace of the spiral accretion stream in the simulated image by identifying local radial maxima in the image in steps of 1\,$\degr$. The trace, shown with white dots, is then also plotted on top of the F187N and F480M images. The correspondence with the observed spiral-like signal in the F187N image and with the inner arm of the tentative fork seen in the F480M image is remarkable (better seen in Fig.~\ref{fig:FinalImages}e and f)

The arm-like signal identified and characterized in \citet{Juillard2022} is also shown with orange dots in Fig.~\ref{fig:SpiralTrace}c. This corresponds to the trace inferred from a 2021 H-band VLT/SPHERE dataset
\citep[ESO Program 60.A-9801; see more details in][]{Juillard2022}. We also note a good agreement with the outer arm %of the tentative fork 
observed in the F480M image, suggesting that it traces the same feature as observed in VLT/SPHERE images. The coarse angular resolution of the image and the different wavelength of the observation (compared to prior ground-based images in which the feature is seen) prevents a meaningful proper motion analysis of the spiral feature similar to what was done in \citet{Juillard2022}, however.

\section{Constraints on additional planets}\label{sec:contrast_curves}

\subsection{Signal-to-noise ratio maps}

    \begin{figure*}
    \centering
    \includegraphics[width=\textwidth]{SNRmaps_F480M.pdf}
    \caption{S/N maps of the F480M images obtained after roll subtraction and after subtraction of the disk and planets using the optimal parameters found in Appendix~\ref{sec:NEGFD} and \ref{sec:NEGFC_optim}. The left (middle) panel shows the S/N map after subtracting the disk and planet $c$ ($b$) alone, and the right panel shows the S/N map after removing the disk and both planets. Planets b and $c$ and the spiral-like feature are found at S/N values of $\sim$16.4, $\sim$13.9, and $\sim$5.1, respectively. The color bars cover S/N values ranging from -5 up to the maximum S/N value of each respective map.}
    \label{fig:SNRmaps}
    \end{figure*}
    
Figure \ref{fig:SNRmaps} shows the S/N maps obtained with roll subtraction on the image cube from which the optimal radiative disk model and protoplanets $b$ and c were all subtracted with their optimal parameters found with NEGFD (Appendix~\ref{sec:NEGFD}) and NEGFC (Appendix~\ref{sec:NEGFC_optim}), respectively.
Because the protoplanets share similar projected radii, the S/N maps are shown without planet $c$ ($b$) to evaluate the significance of planet $b$ ($c$) in the left (middle) panel. Both planets are removed in the right panel to assess the S/N of the residual signals apart from the planets. We do not detect any additional planet candidate in the outer disk. Only the spiral-like feature located in the outer wake of planet $c$ stands out at S/N$\gtrsim5$. 

\subsection{Mass sensitivity curves}

    \begin{figure*}
    \centering
    \includegraphics[width=0.495\textwidth]{FigDa.pdf}
    \includegraphics[width=0.495\textwidth]{FigDb.pdf}
    \caption{5 $\sigma$ contrast limits (left) and mass sensitivity (right) for the F480M observations. We show both PCA-roll and TRAP \citep{Samland2021} contrast limits and convert them into masses with a range of evolutionary models. The uncertainty on the mass sensitivity is propagated based on the uncertainty of the host star.}
    \label{fig:contrast_mass_limits}
    \end{figure*}

Fig.~\ref{fig:contrast_mass_limits} (left) shows the contrast curves achieved with PCA-roll (Appendix~\ref{sec:AltAlgos}) and TRAP \citep{Samland2021} on the image cube obtained after subtracting the optimal disk and planet models determined with NEGFD and NEGFC, respectively. A wedge was defined to encompass PA values between 10 and 160\,$\degr$ when calculating the contrast curves to avoid any significant disk residuals, including the arm-like feature, 
from biasing the contrast estimation.

We converted the achieved contrasts into mass sensitivities using the ATMO2020 models \citep{Phillips2020} with both equilibrium and nonequilibrium chemistry, and the BEX models \citep{Linder2019} assuming an age of 5.4$\pm$1\,Myr \citep{Muller2018}. These mass limits are shown in Fig.~\ref{fig:contrast_mass_limits} (right), in which the shaded region represents the 1$\sigma$ uncertainty on the mass limits due to the uncertainty in target age. The ATMO2020 evolutionary models cover masses above 0.524$M_J$, and the BEX models cover masses below 2$M_J$. There is some scatter in the calculated mass sensitivity based on the different spectra for each model assumption; models at this young age are also highly sensitive to initial conditions.
The curves show that we would have been sensitive to planets with a mass $\lesssim 2 M_J$ ($\lesssim$0.8$M_J$) at $\sim$0.6$\arcsec$ (resp.~$\gtrsim$1$\arcsec$) separation from the star, according to the BEX and ATMO (in chemical equilibrium) models; we note that we reached the bottom of the ATMO grid of models (in chemical equilibrium) in terms of planet mass. In case of nonequilibrium chemistry in the atmosphere, planets as massive as 2-4 $M_J$ may still be present, if unseen, in the outer disk.
Even higher-mass giant planets may be hidden in the outer disk \vc{if} their signal is extinct. Nonetheless, multi-Jovian mass giant planets would likely also have affected the gas density profile, which ALMA reveals to extend beyond $1\farcs5$ (i.e., beyond 170~au) in radius without any additional gaps than the gap that is carved by protoplanets $b$ and $c$ \citep{Keppler2019}.

% plain-text version of title in brackets to avoid PDF warnings
\section[The F187N excess for planet c]{F187N excess for planet {\sf c}}\label{sec:F187N_excess}

%A first potential cause is 
All flux estimates shown in Figs.~\ref{fig:spec_b} and \ref{fig:spec_c} except for those from the NIRCam rely on either a model of the star plus inner disk or on absolute (spectro-)photometry of the star measured at a different epoch than the epoch of the observation that allowed the contrast measurement of the planet with respect to the star 
for the scaling of contrasts to absolute fluxes. The stellar variability plus unresolved inner-disk flux \citep{Casassus2022, Perotti2023} can thus result in significant differences between the actual photometry and the assumed photometry for the star, and may therefore accordingly affect the protoplanet flux. To illustrate the amplitude of this bias, we multiplied our measured contrasts for $b$ and $c$ by the F187N stellar flux inferred based on the 0.7-2.7$\upmu$m SpeX spectrum of PDS~70 \citep[][]{Long2018} used in other works. We show the results with light blue error bars in Figs.~\ref{fig:spec_b} and \ref{fig:spec_c}. This may thus partly account for the observed discrepancy.

Another source of misestimation for the flux of planet $c$ is the bright outer-disk edge for measurements leveraging 
% cause for the discrepancy may be that literature spectro-photometry of $c$ extracted %based on data %acquired and reduced with an 
% using 
angular differential imaging %strategy 
\citep{Marois2006}, and performed without prior disk model subtraction from the images.
% and without outer disk model subtraction %nor sufficient angular resolution to separate the contamination from the outer disk edge 
% \citep[i.e., all points but the Keck/NIRC2 L' photometry and GRAVITY K-band spectrum][]{Wang2020, Wang2021} are underestimated when using a classical NEGFC or forward-modeling approach, due to self- and over-subtraction effects of the bright disk. %associated to the bright disk being present in the model stellar halo that is subtracted. 
Bright disk signals captured in the model stellar PSF %images %that are subtracted 
lead to a pair of negative traces encasing the positive trace of the disk in PSF-subtracted images, similar to what can be seen in the roll-subtraction images (Fig.~\ref{fig:AltAlgos}a and e). This significantly dampens any other signal in the negative trace (e.g., planet $c$ and the accretion stream in Fig.~\ref{fig:AltAlgos}a). 
Classical NEGFC (with the sum figure of merit) or other point-source forward-modeling approaches are not tailored to properly account for self- and oversubtraction effects associated with the bright disk and may underestimate the point-source flux when it is located in a very negative disk trace like this. %associated to the bright disk being present in the model stellar halo that is subtracted. 
The exact amplitude of this bias is difficult to assess as it depends on the disk scattering phase function (stronger effect close to PA$_b$ where the disk is brightest), the number of principal components used for PCA-based reductions \citep[see e.g., Fig.~2 in][]{Juillard2022}, and the wavelength of the observation because the %scattered-light flux of the 
disk is brighter, hence the effect stronger, at shorter wavelengths. %(i.e., affecting the most the SPHERE/IFS spectrum shortward of 1.66\,$\upmu$m), 
%where scattered stellar light of the disk, hence the associated negative trace, is stronger. 
If the SPHERE/IFS measurements (0.9--1.6\,$\upmu$m) are underestimated, the best-fit models presented in \citet{Wang2021} may also underpredict the F187N flux, hence inflate our reported 1.87$\upmu$m excess. % at 1.87 $\upmu$m. %-- in particular these models could be much less extinct (i.e., $A_V < 18$ mag for the BT-SETTL and EXOREM models) or correspond to a different best-fit DRIFT-PHOENIX model; 
%Nonetheless, according to fake companion injection tests in \citet{Wang2021}, correction of this bias should be mostly accounted for in the SPHERE/IFS measurements, and not lead to deviations significantly larger than the 
%reported uncertainties.

Another possibility is the inclusion of bright circumplanetary signals that are resolved in ground-based observations, but are unresolved in the lower angular resolution images of our observations, hence contributing only to our NIRCam measurements. However, the difference in angular resolution between the longest wavelengths of the SPHERE/IFS observations and our JWST/NIRCam images is minor, with an estimated FWHM about $\sim$70\% smaller for the former compared to the latter. This hypothesis therefore appears to be unlikely to account alone for the excess of a factor of $3 \pm 1$ measured in the F187N compared to atmospheric model predictions.  
%hence would not be able to Bright circumplanetary signals that are part of the accretion stream which are resolved in ground-based images of planet $c$ surround. 

%Yet another hypothesis is the lack of a water vapour absorption band at 1.8--2.0\,$\upmu$. This appears unlikely given the decreasing fluxes measured by GRAVITY with increasing wavelengths at 2.2--2.4\,$\upmu$, likely attributable to strong water vapour absorption. 

Alternatively, part of the measured F187N excess may be assigned to the presence of significant Pa-$\alpha$ line emission. This hypothesis is discussed in more details in the main text (Sec.~\ref{sec:photometry}).

%(iii) presence of a scattered light contribution from small dust in the unresolved circumplanetary disk of planet $c$, which is getting more important as the planet is approaching $PA_{\rm b}$; (iv) 
%If none of the above causes fully accounts for the observed excess, the residual difference may be due to 

\section{Schematic representation of the system}

    \begin{figure}[h]
    \centering
    \includegraphics[width=0.495\textwidth]{FigG_v2.png}
    \caption{Schematic summary of our proposed interpretation for the main features detected in our NIRCam observation, including an accretion stream feeding the circumplanetary disk of planet $c$ and a protoplanet candidate $d$ at the edge of the inner disk.}
    \label{fig:schema}
    \end{figure}
    
In Figure~\ref{fig:schema} we summarize our knowledge about and interpretation of the different features of the PDS~70 system, drawing on \citet{Keppler2018, Keppler2019}, %\citet{Keppler2019}, 
\citet{Mesa2019a}, \citet{Isella2019}, \citet{Benisty2021}, \citet{Casassus2022}, \citet{Wang2021}, and this work.


\end{appendix}
\end{document}
